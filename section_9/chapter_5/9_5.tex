\documentclass[letterpaper, 12pt]{article}
\usepackage{amsmath, amssymb}
\usepackage{mathtools}

\newtheorem{theorem}{Theorem}[section]
\newtheorem{lemma}[theorem]{Lemma}
\newtheorem{proposition}[theorem]{Proposition}
\newtheorem{corollary}[theorem]{Corollary}

\newenvironment{proof}[1][Proof]{\begin{trivlist}
\item[\hskip \labelsep {\bfseries #1}]}{\end{trivlist}}
\newenvironment{definition}[1][Definition]{\begin{trivlist}
\item[\hskip \labelsep {\bfseries #1}]}{\end{trivlist}}
\newenvironment{example}[1][Example]{\begin{trivlist}
\item[\hskip \labelsep {\bfseries #1}]}{\end{trivlist}}
\newenvironment{remark}[1][Remark]{\begin{trivlist}
\item[\hskip \labelsep {\bfseries #1}]}{\end{trivlist}}

\newcommand{\qed}{\quad \blacksquare}
\newcommand{\keyword}[1]{\textbf{#1}}
\newcommand{\then}{\rightarrow}
\newcommand{\bithen}{\leftrightarrow}
\DeclareMathOperator{\lcm}{lcm}
\DeclareMathOperator{\Dom}{Dom}

\newcommand{\N}{\mathbb{N}}
\newcommand{\Z}{\mathbb{Z}}
\newcommand{\Q}{\mathbb{Q}}
\newcommand{\R}{\mathbb{R}}
\newcommand{\C}{\mathbb{C}}
\newcommand{\0}{\emptyset}
\newcommand{\power}{\mathcal{P}}

\newcommand{\divby}{\; \vdots \;}

\title{MATH 381 Section 9.5}
\author{Prof. Olivia Dumitrescu}
\date{12 April 2024}

\begin{document}
    \maketitle
    \begin{definition}
        A relation $R$ on a set $A$ is a relation from $A$ to $A$.
        \[xRy \leftrightarrow (x, y) \in R\]
    \end{definition}
    A matrix can be defined which is associated with a relation $R$
    \[a_{ij} = \begin{cases}
        1 & (i, j) \in R \\
        0 & (i, j) \notin R
    \end{cases}\]
    The relation $R$ can be interpreted as a directed graph. The vertices of the directed graph 
    consist of the points of $A$.
    \begin{enumerate}
        \item $\overline{xy}$ is a directed edge iff $(x, y) \in R$
        \item $x \circlearrowright$ is a directed loop iff $(x, x) \in R$
    \end{enumerate}
    \section*{Section 9.5 Equivalence Relations}
    \begin{definition}
        A relation $R$ on a set $A$ is an equivalence relation if it satisfies:
        \begin{enumerate}
            \item reflexivity
            \[\forall a \in A (a \overset{R}{\sim} a)\]
            \item symmetry
            \[\forall a, b \in A (a \overset{R}{\sim} b \then b \overset{R}{\sim} a)\]
            \item transitivity
            \[\forall a, b, c \in A 
            (a \overset{R}{\sim} b \wedge b \overset{R}{\sim} c \then a \overset{R}{\sim} c)\]
        \end{enumerate}
    \end{definition}
    \begin{remark}
        None of the three conditions for an equivalence relation is redundant
    \end{remark}
    \begin{definition}
        Two elements $a$ and $b$ that are related by an equivalence relation are called equivalent 
        (denoted by $a \sim b$ if $a$ and $b$ are equivalent elements with respect to some 
        equivalence.)
    \end{definition}
    \begin{example}
        Let $R$ be a relation. \\
        $a \overset{R}{\sim} b \iff$ either $a=b$ or $a=-b$. \\
        Prove that this is an equivalence relation.
        \begin{enumerate}
            \item if $a \overset{R}{\sim} b$ then also $b \overset{R}{\sim} a$\\
            i.e. if $a = b$ or $a=-b$ then either $b=a$ or $b=-a$.
            \item $\forall a \in A (a \overset{R}{\sim} a)$ \\
            i.e. either $a=a$ or $a=-a$
            \item $\forall a, b, c \in A$
            \[a \overset{R}{\sim} b \wedge b \overset{R}{\sim} c \then a \overset{R}{\sim} c\]
            \[\begin{cases}
                a = b \vee a = -b \\
                b=c \vee b=-c
            \end{cases} \then a = c \vee a = -c\]
        \end{enumerate}
    \end{example}
    \begin{example}
        $A = \R$ \\
        \[a \overset{R}{\sim} b \iff a-b \in \Z\]
        Is this an equivalence relation?
    \end{example}
    \begin{example}
        Is $a \mid b$ an equivalence relation? No
    \end{example}
    \begin{example}
        Prove that congruence modulo $m$ is an equivalence relation.
        \[m \mid a - b \then a \overset{R}{\sim} b\]
    \end{example}
    \begin{example}
        $A = \R \forall a, b \in \R$ \\
        Define $a \overset{R}{\sim} b \iff |a-b| < 1$ \\
        Prove that it is not an equivalence relation.
    \end{example}
    \begin{definition}
        $R$ is an equivalence relation on a set $A$. \\
        The set of elements that are related to $a \in A$ is called the \keyword{equivalence class} 
        of $a$. The equivalence class of $a \in A$ with respect to relation $R$ can be written as 
        \[[a]_R = \{s \in A \mid a \overset{R}{\sim} s\}\]
    \end{definition}
    \begin{theorem}
        If $R$ is an equivalence relation on a set $A$ and taking $a, b \in A$, then the following 
        two statements are equivalent
        \begin{enumerate}
            \item $a \overset{R}{\sim} b$
            \item $[a]_R = [b]_R$
            \item $[a]_R \cap [b]_R \ne \emptyset$
        \end{enumerate}
    \end{theorem}
    \begin{corollary}
        This theorem implies 
        \[A = \bigsqcup_{a \in A} [a]_R\]
        \begin{itemize}
            \item i.e. if $a \overset{R}{\nsim} b$ then $[a]_R \cap [b]_R = \emptyset$
            \item if $a \overset{R}{\sim} b$ then $[a]_R = [b]_R$
        \end{itemize}
    \end{corollary}
    \begin{theorem}
        Let $R$ be an equivalence relation on a set $S$. The equivalence classes with respect to 
        $R$ form a partition of $S$. \\
        Conversely, given any partition $\{A_i \mid i \in I\}$ of subsets of $S$ then there exists 
        an equivalence relation $R$ that has these subsets as an equivalence class.
    \end{theorem}
    \begin{example}
        of partition
        \begin{align*}
            S &= \{1,2,3,4,5,6\} \\
            A_1 &= \{1,2,3\} \\
            A_2 &= \{4,5\} \\
            A_3 &= \{6\} \\
            S &= A_1 \sqcup A_2 \sqcup A_3
        \end{align*}
    \end{example}
\end{document}
