\documentclass[letterpaper, 12pt]{article}
\usepackage{amsmath, amssymb}

\newtheorem{theorem}{Theorem}[section]
\newtheorem{lemma}[theorem]{Lemma}
\newtheorem{proposition}[theorem]{Proposition}
\newtheorem{corollary}[theorem]{Corollary}

\newenvironment{proof}[1][Proof]{\begin{trivlist}
\item[\hskip \labelsep {\bfseries #1}]}{\end{trivlist}}
\newenvironment{definition}[1][Definition]{\begin{trivlist}
\item[\hskip \labelsep {\bfseries #1}]}{\end{trivlist}}
\newenvironment{example}[1][Example]{\begin{trivlist}
\item[\hskip \labelsep {\bfseries #1}]}{\end{trivlist}}
\newenvironment{remark}[1][Remark]{\begin{trivlist}
\item[\hskip \labelsep {\bfseries #1}]}{\end{trivlist}}

\newcommand{\qed}{\nobreak \ifvmode \relax \else
    \ifdim\lastskip<1.5em \hskip-\lastskip
    \hskip1.5em plus0em minus0.5em \fi \nobreak
    \vrule height0.75em width0.5em depth0.25em\fi}
\newcommand{\keyword}[1]{\textbf{#1}}
\newcommand{\then}{\rightarrow}
\newcommand{\bithen}{\leftrightarrow}

\title{MATH 381 Section 2.1}
\author{Olivia Dumitrescu}
\date{7 February 2024}

\begin{document}
    \maketitle
    \section*{Sets}
    \begin{definition}
        A set is an unordered collection of disjoint objects called elements or members.
        A set is said to contain its elements. \\
        We write $a \in A$ to denote that $a$ is an element of the set $A$. \\
        We write $a \notin A$ to denote that $a$ is not an element of the set $A$.
    \end{definition}
    \begin{example}
        \hfil
        \begin{enumerate}
            \item Set of vowels
            \[V = \{a, e, i, o, u\}\]
            \item odd positive integers less than 10
            \[\{1, 3, 5, 7, 9\}\]
        \end{enumerate}
    \end{example}
    \begin{definition}
        \hfill
        \begin{enumerate}
            \item Two sets are equal if and only if they have the same elements.
            \item A set $A$ is a subset of a set $B$
            \[A \subseteq B (\forall x \in A \then x \in B)\]
        \end{enumerate}
    \end{definition}
    \begin{remark}
        To prove $A = B$ is to prove that $A \subseteq B$ and $B \subseteq A$.
    \end{remark}
    \begin{theorem}
        For any set $S$
        \begin{enumerate}
            \item $\emptyset \subset S$
            \item $S \subseteq S$
        \end{enumerate}
    \end{theorem}
    \begin{definition}
        If there are exactly $n$ distinct elements in set $S$ we call $n$ to be cardinality of $S$
        \[|S| = n\]
        E.g. $|\emptyset| = 0$ \\
        A set is infinite if it has no finite cardinality.
    \end{definition}
    \begin{definition}
        Given a set $S$ the power of $S$ is the set of all subsets of $S$
        \[\mathcal{P}(S) = \{A | A \subseteq S\}\]
    \end{definition}
    \begin{theorem}
        If $S$ is a finite set
        \[|\mathcal{P}(S)| = 2^{|S|}\]
    \end{theorem}
    $\binom{n}{i}$: number of distinct subsets of cardinality $i$ of $S$ (with cardinality $n$)
    \begin{theorem}
        \[\binom{n}{k} = \binom{n}{n - k}\]
        \[n \in \mathbb{N} \text{ and } 0 \le k \le n\]
    \end{theorem}
    \begin{theorem}
        Binomial Theorem
        \[(x + y)^n = \sum_{k = 0}^{n} \binom{n}{k} x^k y^{n - k}\]
    \end{theorem}
    \begin{theorem}
        If $S$ is finite, then $|\mathcal{P}(S)| = 2^{|S|} = 2^n$
    \end{theorem}
    \begin{proof}
        Plug in $x = y = 1$ into the binomial theorem.
        \[2^n = \binom{n}{0} + \binom{n}{1} + \binom{n}{2} + \dots + \binom{n}{n-1} + \binom{n}{n}\]
    \end{proof}
    \subsection*{Cartesian Products of Sets}
    \begin{definition}
        $A, B$ are sets.
        \begin{enumerate}
            \item $A \times B = \{(a, b) \: | \: a \in A, b \in B\}$
            \item consider $A_1, A_2, \dots, A_n$
            \[A_1 \times A_2 \times \dots \times A_n = \{(a_1, a_2, \dots, a_n) \: | \: a_i \in A_i\}\]
        \end{enumerate}
    \end{definition}
    \begin{remark}
        If $A$ and $B$ are finite sets,
        \begin{enumerate}
            \item $|A \times B| = |A| \cdot |B|$
            \[|A_1 \times A_2 \times \dots \times A_n| = |A_1| \cdot |A_2| \cdot \dots \cdot |A_n|\]
            $A_i$ is a finite set
            \item $|P(A_1 \times \dots \times A_n)| = 2^{|A_1 \times \dots \times A_n|}
            = 2^{|A_1| \cdot |A_2| \cdot \dots \cdot |A_n|}$
        \end{enumerate}
    \end{remark}
\end{document}
