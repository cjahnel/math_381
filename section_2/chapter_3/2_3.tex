\documentclass[letterpaper, 12pt]{article}
\usepackage{amsmath, amssymb}

\newtheorem{theorem}{Theorem}[section]
\newtheorem{lemma}[theorem]{Lemma}
\newtheorem{proposition}[theorem]{Proposition}
\newtheorem{corollary}[theorem]{Corollary}

\newenvironment{proof}[1][Proof]{\begin{trivlist}
\item[\hskip \labelsep {\bfseries #1}]}{\end{trivlist}}
\newenvironment{definition}[1][Definition]{\begin{trivlist}
\item[\hskip \labelsep {\bfseries #1}]}{\end{trivlist}}
\newenvironment{example}[1][Example]{\begin{trivlist}
\item[\hskip \labelsep {\bfseries #1}]}{\end{trivlist}}
\newenvironment{remark}[1][Remark]{\begin{trivlist}
\item[\hskip \labelsep {\bfseries #1}]}{\end{trivlist}}

\newcommand{\qed}{\nobreak \ifvmode \relax \else
    \ifdim\lastskip<1.5em \hskip-\lastskip
    \hskip1.5em plus0em minus0.5em \fi \nobreak
    \vrule height0.75em width0.5em depth0.25em\fi}
\newcommand{\keyword}[1]{\textbf{#1}}
\newcommand{\then}{\rightarrow}
\newcommand{\bithen}{\leftrightarrow}

\title{MATH 381 Section 2.3}
\author{Olivia Dumitrescu}
\date{14 February 2024}

\begin{document}
    \maketitle
    \section*{Functions}
    \begin{example}
        Prove that there are $\infty$  many pairs of integer solutions on the cone 
        \[x^2 + y^2 - z^2 = 0\]
    \end{example}
    \begin{definition}
        Let $A, B$ be non-empty sets. \\
        A function $f: A \rightarrow B$ to be an assignment of exactly one element of $B$ to 
        each element of $A$.
        \[f(a) = b\]
        \[A = \text{domain}, B = \text{codomain}, 
        b = \text{image of $a$}, a = \text{preimage of $b$}\]
        $Im f \subseteq B$ is called the range of $f$.
        \[Im f = \{b \in B \mid \exists a \in A \text{ s.t. } f(a) = b\}\]
    \end{definition}
    \begin{example}
        The circle $x^2 + y^2 = 1$ is not a function but the union of 2 functions.
        \[y = \begin{cases}
            \sqrt{1 - x^2} \\
            -\sqrt{1 - x^2}
        \end{cases}\]
    \end{example}
    \begin{example}
        Now take
        \begin{align*}
            f_1 : A &\rightarrow B \\
            f_2 : A &\rightarrow B \quad B \subseteq \mathbb{R}
        \end{align*}
        \begin{align*}
            (f_1 + f_2)(x) &= f_1(x) + f_2(x) \quad \forall x \in A \\
            (f_1 \cdot f_2)(x) &= f_1(x) \cdot f_2(x)
        \end{align*}
    \end{example}
    \begin{definition}
        A function $f: A \rightarrow B$ is \keyword{injective} if and only if 
        $f(a) = f(b) \implies a = b$. \\
        This means that distinct points in the domain have different heights 
        i.e. if $a \ne b$, then $f(a) \ne f(b)$.
    \end{definition}
    \begin{remark}
        To disprove that a function is injunctive, it is enough to find two points of the 
        domain $a \ne b$ so that $f(a) = f(b)$. \\
        If you know its graph, how can you test if a function is injective?
        \begin{enumerate}
            \item $A, B$ are finite sets
            \item Continuous functions \\
            \keyword{Horizontal Line Test}: A function is injective if any horizontal line 
            intersects the graph at at most 1 point.
        \end{enumerate}
    \end{remark}
    \begin{example}
        $y = \sqrt{1 - x^2}$ is not injective because $f(x) = f(-x)$.
    \end{example}
    \begin{definition}
        We say a function $f: A \rightarrow B$ is \keyword{surjective} if the range of $f$ is 
        the codomain $B$.
        \[Im f = \{b \in B \mid \exists a \in A, \; f(a) = b\} = B\]
        To prove a function is surjective:
        \[\forall b \in B \; \exists a \in A \text{ so that } f(a) = b\]
    \end{definition}
    \begin{remark}
        We can always make a function surjective by reducing the codomain.
        \[f: A \rightarrow B \qquad f: A \rightarrow Im f \subseteq B\]
    \end{remark}
    \begin{example}
        To make $f: x \mapsto x^2$ surjective, 
        \[f: \mathbb{R} \rightarrow \mathbb{R}_+\]
        To make $f$ injective, 
        \[f: \mathbb{R}_+ \rightarrow \mathbb{R}\]
    \end{example}
    \begin{definition}
        The function $f: A \rightarrow B$ is \keyword{bijective} if $f$ is both injective and 
        surjective.
    \end{definition}
    \begin{remark}
        Any continuous function can be transformed into a bijective function, but not in a 
        unique way necessarily.
    \end{remark}
    \begin{remark}
        Let $f$ be a bijection
        \[A \xrightarrow{f} B\]
        between finite sets.
        \[\implies |A| = |B|\]
    \end{remark}
    \begin{example}
        Bijection between open and bounded interval and $\mathbb{R}$.
        \[\tan(x) : \mathbb{R} - \{\frac{2k + 1}{2}\pi \mid k \in \mathbb{Z}\} \rightarrow \mathbb{R}\]
        So, $\tan$ is sujective but not injective.
        \[\arctan(x) : \mathbb{R} \rightarrow (\frac{-\pi}{2}, \frac{\pi}{2})\]
        $\arctan$ is injective but not surjective.
    \end{example}
    \begin{definition}
        If $f$ is bijective between $A$ and $B$, then there exists an inverse function $f^{-1}$
        \begin{align*}
            &f: A \rightarrow B \\
            &f: A \rightarrow \text{Range} f = \text{Im} f
        \end{align*}
        \begin{gather*}
            \exists f^{-1} : \text{Im} f \rightarrow A = A\xleftarrow{f^{-1}} \text{Im} f \\
            \text{if } f(a) = b \implies f^{-1}(b) = a \qquad a \in A \wedge b \in B
        \end{gather*}
        \begin{gather*}
            f: A \rightarrow \text{Im} f = C \\
            \forall b \in \text{Im} f \; \exists! a \in A \: (b = f(a))
        \end{gather*}
    \end{definition}
    \begin{proposition}
        \[\begin{cases}
            f: A \rightarrow B \\
            k: \text{Im} f \Rightarrow C
        \end{cases}
        \implies k \circ f : A \rightarrow C\]
    \end{proposition}
    \begin{definition}
        We say that $f : A \rightarrow B$ is invertible if 
        $\exists g = f^{-1} : B \rightarrow A$ so that
        \begin{enumerate}
            \item $f \circ g = Id_B (\forall x \in B \implies f(g(x)) = x)$
            \item $g \circ f = Id_A (\forall x \in A \implies g(f(x)) = x)$
            \[A \xrightarrow{f} B \xrightarrow{g} A \implies g \circ f : A \rightarrow A\]
        \end{enumerate}
    \end{definition}
    \begin{remark}
        Start from a bijective function $f: A \rightarrow B$ then $g: B\rightarrow A$. \\
        Can we find $g$?
        \begin{enumerate}
            \item $f(x) = y$
            \item Solve it for $x$ \quad $x = g(y)$
            \item Interchange $x$ and $y$ \quad $y = g(x) = f^{-1}$
        \end{enumerate}
    \end{remark}
    \begin{proposition}
        $f: A \rightarrow B$ is bijective and continuous $\implies$ the inverse 
        $g: B \rightarrow A$ exists. \\
        The graph of $g$ is obtained from the graph of $f$ by reflecting along $y = x$.
    \end{proposition}
\end{document}
