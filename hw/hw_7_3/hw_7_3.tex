\documentclass[letterpaper, 12pt]{article}
\usepackage{amsmath, amssymb}
\usepackage{graphicx}
\graphicspath{{./images/}}

\newtheorem{theorem}{Theorem}[section]
\newtheorem{lemma}[theorem]{Lemma}
\newtheorem{proposition}[theorem]{Proposition}
\newtheorem{corollary}[theorem]{Corollary}

\newenvironment{proof}[1][Proof]{\begin{trivlist}
\item[\hskip \labelsep {\bfseries #1}]}{\end{trivlist}}
\newenvironment{definition}[1][Definition]{\begin{trivlist}
\item[\hskip \labelsep {\bfseries #1}]}{\end{trivlist}}
\newenvironment{example}[1][Example]{\begin{trivlist}
\item[\hskip \labelsep {\bfseries #1}]}{\end{trivlist}}
\newenvironment{remark}[1][Remark]{\begin{trivlist}
\item[\hskip \labelsep {\bfseries #1}]}{\end{trivlist}}

\newcommand{\qed}{\quad \blacksquare}
\newcommand{\keyword}[1]{\textbf{#1}}
\newcommand{\then}{\rightarrow}
\newcommand{\bithen}{\leftrightarrow}

\newcommand{\N}{\mathbb{N}}
\newcommand{\Z}{\mathbb{Z}}
\newcommand{\Q}{\mathbb{Q}}
\newcommand{\R}{\mathbb{R}}
\newcommand{\C}{\mathbb{C}}
\newcommand{\0}{\emptyset}

\title{MATH 381 HW 7 part 3}
\author{Christian Jahnel}
\date{28 February 2024}

\begin{document}
\maketitle
\begin{enumerate}
\item Let $U$ be a universe and $A \subseteq U$. Define the characteristic function of $A$, 
$\chi_A : U \to \{0, 1\}$, by $\chi_A(x) = \begin{cases}
    1 & \text{if $x \in A$} \\
    0 & \text{if $x \notin A$}
\end{cases}$
\begin{enumerate}
\item Consider $A = [0, 2) \subseteq \R$. Sketch the graph of $\chi_A$.
\begin{center}
\includegraphics[scale=0.3]{graph.jpeg}
\end{center}
\item Again consider $A = [0, 2) \subseteq \R$. Determine the sets $(\chi_A)^{-1}(1)$ and 
$(\chi_A)^{-1}(2)$.
\begin{gather*}
    (\chi_A)^{-1}: \{0, 1\} \to U \\
    (\chi_A)^{-1}(1) = \{(\chi_A)^{-1}(\chi_A(x)) \mid x \in A\} = \{x \mid x \in A\} = [0, 2)\\
    (\chi_A)^{-1}(2) = \emptyset \because 2 \notin \{0, 1\} = \text{domain of } (\chi_A)^{-1}
\end{gather*}
\end{enumerate}
\item Give examples of sets $S, T \subseteq \R$ and a function $f: \R \to \R$ for which 
$f(S \cap T) \ne f(S) \cap f(T)$. Clearly justify why your sets and function satisfy the property.
\begin{align*}
    f(S) &= \{f(x) \mid x \in S\} \\
    f(T) &= \{f(x) \mid x \in T\} \\
    f(S \cap T) &= \{f(x) \mid x \in S \cap T\} = \{f(x) \mid x \in S \wedge s \in T\} \\
    f(S) \cap f(T) &= \{f(x) \mid x \in S\} \cap \{f(x) \mid x \in T\}
\end{align*}
\begin{align*}
    S &= \{0\} \\
    T &= \{1\} \\
    S \cap T &= \0 \\
    f &: x \mapsto 1 \\
    f(S) &= \{1\} \\
    f(T) &= \{1\} \\
    f(S \cap T) &= \0 \\
    f(S) \cap f(T) &= \{1\} \\
    \therefore f(S \cap T) &\ne f(S) \cap f(T) \qed
\end{align*}
\begin{flushleft}
    Essentially, this phenomenon arises from the fact that the function is not injective since 
    it maps both $x = 0$ and $x = 1$ to $f(x) = 1$. Since the two singleton sets are disjoint, 
    their intersection is the empty set, and thus so is the image. However, the images of both 
    sets are not disjoint, and therefore their intersection is a singleton set.
\end{flushleft}
\pagebreak
\item Let $f: \R^2 \to \R$ be given by $f(x, y) = x - y$. Determine, with proofs, whether $f$ 
is one-to-one, onto, both, or neither. Be sure to fully justify both your statement of 
one-to-one and your statement of onto.
\begin{enumerate}
\item $P: \text{$f$ is an injective (i.e. one-to-one) function}$
\[\begin{split}
    P &\equiv \forall (x_1, y_1), (x_2, y_2) \in \R^2(f(x_1, y_1) = f(x_2, y_2) 
    \then (x_1, y_1) = (x_2, y_2)) \\
    &\equiv \forall (x_1, y_1), (x_2, y_2) \in \R^2((x_1, y_1) \ne (x_2, y_2) 
    \then f(x_1, y_1) \ne f(x_2, y_2))
\end{split}\]
Suppose $(x_1, y_1) \ne (x_2, y_2)$ and let $x_1 = y_1$ and $x_2 = y_2$.
\begin{gather*}
\begin{aligned}
    f(x_1, y_1) = x_1 - y_1 = x_1 - x_1 = 0 \\
    f(x_2, y_2) = x_2 - y_2 = x_2 - x_2 = 0
\end{aligned}
\implies f(x_1, y_1) = f(x_2, y_2)
\end{gather*}
\[\begin{split}
    &\therefore \exists (x_1, y_1), (x_2, y_2) \in \R^2 ((x_1, y_1) \ne (x_2, y_2) 
    \then f(x_1, y_1) = f(x_2, y_2)) \\
    &\implies \neg P
\end{split}\]
\item $Q: \text{$f$ is a surjective (i.e. onto) function}$
\[Q \equiv \forall z \in \R \; \exists x, y \in \R^2(f(x, y) = z)\]
Let $z \in \R$.
\begin{align*}
    \text{Fix } y &= 0 \in \R \\
    \implies z = x - y &\iff z = x \\
    z \in \R &\implies x \in \R
\end{align*}
$\therefore Q \qed$
\end{enumerate}
\begin{flushleft}
    Consequently, \textbf{$f$ is a surjective (i.e. onto) function}, but \textbf{it is not an 
    injective (i.e. one-to-one) function}, and thus it cannot be a bijective function. 
    Intuitively, for every value $z$ on the $z$-axis of the 3-D graph of $f$, there must be a 
    one-dimensional (linear) slice through the plane. Consequently, there are an infinite amount 
    of solutions along each line, and therefore it cannot be injective. As for the surjective 
    property, the plane spans all real numbers in its range since it is angled nonparallel to 
    the $x$-axis, as seen by its coefficients.
\end{flushleft}
\item Suppose $f: A \to B$ is one-to-one and $g: B \to C$ is one-to-one. 
Prove $g \circ f: A \to C$ is also one-to-one.
\begin{multline*}
    \forall x_A, y_A \in A(f(x_A) = f(y_A) \then x_A = y_A) \\
    \wedge \forall x_B, y_B \in B(g(x_B) = g(y_B) \then x_B = y_B)
\end{multline*}
\[P: \forall x_A, y_A \in A((g \circ f)(x_A) = (g \circ f)(y_A) \then x_A = y_A)\]
Let $x_A, y_A \in A$ and suppose $(g \circ f)(x_A) = (g \circ f)(y_A)$.
\begin{align*}
    (g \circ f)(x_A) &= (g \circ f)(y_A) \\
    \implies g(f(x_A)) &= g(f(y_A))
\end{align*}
\begin{align*}
    \text{Let } x_B &= f(x_A) \in B \\
    \text{Let } y_B &= f(y_A) \in B
\end{align*}
\begin{align*}
    g(f(x_A)) &= g(f(y_A)) \\
    \implies g(x_B) &= g(y_B) \\
    \implies x_B &= y_B \\
    \implies f(x_A) &= f(y_A) \\
    \implies x_A = y_A
\end{align*}
$\therefore P \qed$
\end{enumerate}
\end{document}
