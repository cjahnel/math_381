\documentclass[letterpaper, 12pt]{article}
\usepackage{amsmath, amssymb}

\newtheorem{theorem}{Theorem}[section]
\newtheorem{lemma}[theorem]{Lemma}
\newtheorem{proposition}[theorem]{Proposition}
\newtheorem{corollary}[theorem]{Corollary}

\newenvironment{proof}[1][Proof]{\begin{trivlist}
\item[\hskip \labelsep {\bfseries #1}]}{\end{trivlist}}
\newenvironment{definition}[1][Definition]{\begin{trivlist}
\item[\hskip \labelsep {\bfseries #1}]}{\end{trivlist}}
\newenvironment{example}[1][Example]{\begin{trivlist}
\item[\hskip \labelsep {\bfseries #1}]}{\end{trivlist}}
\newenvironment{remark}[1][Remark]{\begin{trivlist}
\item[\hskip \labelsep {\bfseries #1}]}{\end{trivlist}}

\newcommand{\qed}{\quad \blacksquare}
\newcommand{\keyword}[1]{\textbf{#1}}
\newcommand{\then}{\rightarrow}
\newcommand{\bithen}{\leftrightarrow}
\newcommand{\naturals}{\mathbb{N}}
\newcommand{\integers}{\mathbb{Z}}
\newcommand{\rationals}{\mathbb{Q}}
\newcommand{\reals}{\mathbb{R}}
\newcommand{\complex}{\mathbb{C}}

\title{MATH 381 Homework 6 part 2}
\author{Christian Jahnel}
\date{21 February 2024}

\begin{document}
\maketitle
\begin{enumerate}
\item Show that any integer is a multiple of 5 if and only if it square is a multiple of 5.
\begin{flushleft}
    $P(n)$: ``$n$ is a multiple of 5'' $\equiv \exists k \in \mathbb{Z} (n = 5k)$ \\
    $Q(n)$: ``$n^2$ is a multiple of 5'' $\equiv \exists m \in \mathbb{Z} (n^2 = 5m)$ 
\end{flushleft}
Suppose $P(n)$.
\begin{gather*}
    n = 5k \quad k \in \mathbb{Z} \\
    n^2 = (5k)^2 = (5^2 k^2) = 5(5k^2) = 5m \quad m = 5k^2 \in \mathbb{Z}
\end{gather*}
$\therefore Q(n)$ \\
Suppose $Q(n)$.
\begin{gather*}
    n^2 = 5m \quad m \in \mathbb{Z} \\
    \sqrt{n^2} = \sqrt{5m} = \sqrt{5}\sqrt{m}
\end{gather*}
Since $n^2$ is a square of an integer:
\begin{align*}
    & \sqrt{n^2} \in \mathbb{Z}_+ \\
    &\implies \sqrt{5}\sqrt{m} \in \mathbb{Z}_+ \\
    &\implies \sqrt{5}\sqrt{5}\sqrt{m/5} \in \mathbb{Z}_+ \\
    &\implies 5\sqrt{m/5} \in \mathbb{Z}_+ \\
    &\implies \sqrt{m/5} \in \mathbb{Z}_+ \because 5 \in \mathbb{Z}_+ \\
    &\therefore \sqrt{n^2} = n = 5k \qquad k = \sqrt{m/5} \in \mathbb{Z}_+
\end{align*}
$\therefore P(n) \qed$
\item Using the definition $|x| = \begin{cases}
    x & \text{if } x \ge 0 \\
    -x & \text{if } x < 0
\end{cases}$, 
prove that for all real numbers $x$ and $y$, $|x| \le y$ if and only if $-y \le x \le y$.
\begin{flushleft}
    Suppose $|x| \le y$.
\end{flushleft}
\begin{gather*}
    |x| \le y = \begin{cases}
        x \le y \quad \text{ if } x \ge 0 \\
        -x \le y \quad \text{ if } x < 0
    \end{cases} = \begin{cases}
        x \le y \quad \text{ if } x \ge 0 \\
        x \ge -y \quad \text{ if } x < 0
    \end{cases}
\end{gather*}
\begin{align*}
    &\iff (x \le y \wedge x \ge 0) \vee (x \ge -y \wedge x < 0) \\
    &\iff (0 \le x \le y) \vee (-y \le x < 0) \\
    &\iff x \in [0, y] \vee x \in [-y, 0) \\
    &\iff x \in [0, y] \cup [-y, 0) \\
    &\iff x \in [-y, y] \\
    &\iff -y \le x \le y
\end{align*}
$\therefore |x| \le y \leftrightarrow -y \le x \le y \qed$
\pagebreak
\item Prove that if $ab > 0$ and $bc < 0$, then $ax^2 + bx + c = 0$ has two real solutions. \\
\[x = \frac{-b \pm \sqrt{b^2 - 4ac}}{2a}\]
The quadratic equation has two real solutions when $b^2 - 4ac > 0 \iff b^2 > 4ac$.
\begin{gather*}
    (ab > 0) \then (a > 0 \wedge b > 0) \vee (a < 0 \wedge b < 0) \\
    (bc < 0) \then (b < 0 \wedge c > 0) \vee (b > 0 \wedge c < 0)
\end{gather*}
\begin{flushleft}
    There are two possibilities, since $a$ and $c$ are dependent on the value of $b$ to fulfill 
    their inequalities. \\
    If $b$ is negative, then $a$ must be negative as well, and $c$ must be positive. \\
    If $b$ is positive, then $a$ must be positive as well, and $c$ must be negative.
\end{flushleft}
\begin{gather*}
    (b > 0) \then (a > 0 \wedge c < 0) \\
    (b < 0) \then (a < 0 \wedge c > 0)
\end{gather*}
Assume $b < 0$.
\begin{gather*}
    b^2 > 0 \\
    a < 0 \implies 4ac < 0 \\
    \therefore b^2 > 4ac
\end{gather*}
Assume $b > 0$.
\begin{gather*}
    b^2 > 0 \\
    c < 0 \implies 4ac < 0 \\
    \therefore b^2 > 4ac
\end{gather*}
\[\therefore (ab > 0 \wedge bc < 0) \then (b^2 - 4ac > 0) \qed\]
\begin{flushleft}
    Therefore, $ax^2 + bx + c = 0$ has two real solutions if $ab > 0$ and $bc < 0$.
\end{flushleft}
\pagebreak
\item Show that there is a three-digit number less than 400 with distinct digits such that 
the sum of the digits is 17 and the product of the digits is 108. Is there a unique such number? 
Explain.
\begin{gather*}
    n = 100n_2 + 10n_1 + 1n_0 < 400 \\
    n_2 + n_1 + n_0 = 17 \\
    n_2 \cdot n_1 \cdot n_0 = 108 \\
    108 = 12 \cdot 9 = 3 \cdot 4 \cdot 3^2 = 2^2 \cdot 3^3 \\
    n_1, n_2, n_3 \ne 0 \because n_2 \cdot n_1 \cdot n_0 = 108 \ne 0
\end{gather*}
\begin{center}
    \begin{tabular}{c | c | c | c | c}
        $i$ & $j$ & $2^i$ & $3^j$ & $2^i \cdot 3^j$ \\
        0 & 0 & 1 & 1 & *1 \\
        0 & 1 & 1 & 3 & *3 \\
        0 & 2 & 1 & 9 & *9 \\
        0 & 3 & 1 & 27 & 27 \\
        1 & 0 & 2 & 1 & *2 \\
        1 & 1 & 2 & 3 & *6 \\
        1 & 2 & 2 & 9 & 18 \\
        1 & 3 & 2 & 27 & 54 \\
        2 & 0 & 4 & 1 & *4 \\
        2 & 1 & 4 & 3 & 12 \\
        2 & 2 & 4 & 9 & 36 \\
        2 & 3 & 4 & 27 & 108 \\
    \end{tabular}
\end{center}
\begin{gather*}
    n_1, n_2, n_3 \in \{1, 2, 3, 4, 6, 9\} \\
    6 + 3 + 4 = 13 < 17 \implies (n_1 = 9) \vee (n_2 = 9) \vee (n_3 = 9) \\
    17 - 9 = 8 = 6 + 2 \\
    n_1, n_2, n_3 \in \{2, 6, 9\} \\
    n_2 \ne 9 \because n \ge 900 > 400 \\
    n_2 \ne 6 \because n \ge 600 > 400 \\
    \implies n_2 = 2
    (n_2 = 2 \wedge n_1 = 6 \wedge n_0 = 9) \then (n = 269) \\
    (n = 269) \then (n < 400) \\
    \therefore (n_2 = 2 \wedge n_1 = 6 \wedge n_0 = 9) \then (n < 400)
    (n_2 = 2 \wedge n_1 = 9 \wedge n_0 = 6) \then (n = 296) \\
    (n = 296) \then (n < 400) \\
    \therefore (n_2 = 2 \wedge n_1 = 9 \wedge n_0 = 6) \then (n < 400) \qed
\end{gather*}
\begin{flushleft}
    \textbf{Yes, there is such a number}. One such number is 269. Its digits were found by finding 
    combinations of factors in the prime factorization in 108 that could possibly add to 17. 
    9 must be a digit to reach 17 as a sum, and the only other factors that add to 8 are 2 and 
    6. Since 6 and 9 as the first digit would make the number greater than 400, 2 must be the 
    first digit, and a possible number was revealed to be 269 which is indeed less than 400. 
    However, another possible number that fits the restrictions is 296, which is formed by 
    interchanging the last two digits. Therefore, there does exist a number, \keyword{but it 
    is not unique}, as there are two possible numbers that fit the requirements.
\end{flushleft}
\pagebreak
\item Without trying to evaluate these numbers, show that the product of two of the numbers 
$65^{1000} - 8^{2001} + 3^{177}$, $79^{2121} - 9^{2399} + 2^{2001}$, and $24^{4493} + 5^{8192} 
+ 7^{1777}$ is non-negative. \\
\begin{gather*}
    n_1 = 65^{1000} - 8^{2001} + 3^{177} \\
    n_2 = 24^{4493} + 5^{8192} + 7^{1777}
\end{gather*}
\begin{gather}
    (n_1 * n_2 > 0) \leftrightarrow (n_1 > 0 \wedge n_2 > 0) \vee (n_1 < 0 \wedge n_2 < 0) \\
    n_2 > 0 = T \because (24^{4493} > 0) \wedge (5^{8192} > 0) \wedge (7^{1777} > 0) \\
    \therefore (n_1 < 0 \wedge n_2 < 0) = F, \; (n_1 > 0 \wedge n_2 > 0) \equiv (n_1 > 0) \\
    \implies (n_1 * n_2 > 0) \leftrightarrow (n_1 > 0)
\end{gather}
\begin{align}
    65^{1000} > 8^{2001} &\implies 65^{1000} + 3^{177} > 8^{2001} \\
    &\iff 65^{1000} - 8^{2001} + 3^{177} > 0
\end{align}
\begin{flushleft}
    $p$: $65^{1000} - 8^{2001} + 3^{177} > 0$ \\
    $q$: $65^{1000} + 3^{177} > 8^{2001}$ \\
    $r$: $65^{1000} > 8^{2001}$ \\
    $s$: $(n_1 * n_2 > 0)$
\end{flushleft}
\begin{gather*}
    r \then q \qquad \text{(From equation 5)} \\
    q \leftrightarrow p \qquad \text{(From equation 6)} \\
    p \leftrightarrow s \qquad \text{(From equation 4)} \\
    \therefore r \then s
\end{gather*}
\begin{align*}
    \begin{split}
        & 65^{1000} > 8^{2001} \\
        &= (64 + 1)^{1000} > 8^{2(1000.5)} \\
        &= (64 + 1)^{1000} > (8^2)^{1000.5} \\
        &= (64 + 1)^{1000} > (64)^{1000.5} \\
        &= (64 + 1)^{1000} > 64^{0.5}(64)^{1000} \\
        &= (64 + 1)^{1000} > 8(64)^{1000} \\
        &= \sqrt{65}(64 + 1)^{999.5} > 8(64)^{1000} \\
        &= \sqrt{65}(64 + 1)^{999.5} > 8(64)^{1000} \\
    \end{split}
\end{align*}
\begin{gather*}
    \sqrt{65} > \sqrt{64} = 8 \\
    \therefore (8(64 + 1)^{999.5} > 8(64)^{1000}) \then (\sqrt{65}(64 + 1)^{999.5} > 8(64)^{1000})
\end{gather*}
\begin{align*}
    \begin{split}
        & 8(64 + 1)^{999.5} > 8(64)^{1000} \\
        &\iff (64 + 1)^{999.5} > (64)^{1000} \\
        &\iff 999.5\ln(65) > 1000\ln(64) \\
        &\iff \frac{\ln(65)}{\ln(64)} > \frac{1000}{999.5} \quad [1.003727969 > 1.00050025] \qed
    \end{split}
\end{align*}
\begin{flushleft}
    Since we have found a demonstrably true statement ($\frac{\ln(65)}{\ln(64)} > \frac{1000}{999.5}$) 
    and it is logically equivalent to $8(64 + 1)^{999.5} > 8(64)^{1000}$, this statement must 
    be true, which implies that $\sqrt{65}(64 + 1)^{999.5} > 8(64)^{1000}$ is true. Since this 
    statement is ultimately equivalent to our original $r$, $r$ must be true. Finally, we showed 
    that the biconditional statement $r \leftrightarrow s$ is true, so $s$ our original 
    statement that the product of the two numbers we chose is positive is also true.
\end{flushleft}
\end{enumerate}
\end{document}
