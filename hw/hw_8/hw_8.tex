\documentclass[letterpaper, 12pt]{article}
\usepackage{amsmath, amssymb}
\usepackage{enumitem}

\newtheorem{theorem}{Theorem}[section]
\newtheorem{lemma}[theorem]{Lemma}
\newtheorem{proposition}[theorem]{Proposition}
\newtheorem{corollary}[theorem]{Corollary}

\newenvironment{proof}[1][Proof]{\begin{trivlist}
\item[\hskip \labelsep {\bfseries #1}]}{\end{trivlist}}
\newenvironment{definition}[1][Definition]{\begin{trivlist}
\item[\hskip \labelsep {\bfseries #1}]}{\end{trivlist}}
\newenvironment{example}[1][Example]{\begin{trivlist}
\item[\hskip \labelsep {\bfseries #1}]}{\end{trivlist}}
\newenvironment{remark}[1][Remark]{\begin{trivlist}
\item[\hskip \labelsep {\bfseries #1}]}{\end{trivlist}}

\newcommand{\qed}{\quad \blacksquare}
\newcommand{\keyword}[1]{\textbf{#1}}
\newcommand{\then}{\rightarrow}
\newcommand{\bithen}{\leftrightarrow}

\newcommand{\N}{\mathbb{N}}
\newcommand{\Z}{\mathbb{Z}}
\newcommand{\Q}{\mathbb{Q}}
\newcommand{\R}{\mathbb{R}}
\newcommand{\C}{\mathbb{C}}
\newcommand{\0}{\emptyset}

\title{MATH 381 HW 8}
\author{Christian Jahnel}
\date{10 March 2024}

\begin{document}
\maketitle
\begin{enumerate}
\item Prove that $1^3 + 2^3 + \dots + n^3 = \dfrac{n^2(n+1)^2}{4}$ for all positive integers n.
\begin{enumerate}
\item Basis step:
\begin{align*}
    P(1): &\quad 1^3 = \frac{1^2(1+1)^2}{4} &\iff 1 &= 1 \\
    P(2): &\quad 1^3 + 2^3 = \frac{2^2(2+1)^2}{4} &\iff 9 &= 9 \\
    P(3): &\quad 1^3 + 2^3 + 3^3 = \frac{3^2(3+1)^2}{4} &\iff 36 &= 36
\end{align*}
\item Induction step; Assume $P(k)$.
\begin{align*}
    1^3 + 2^3 + \dots + k^3 &= \frac{k^2(k+1)^2}{4} \\
    \implies 1^3 + 2^3 + \dots + k^3 + (k+1)^3 &= \frac{k^2(k+1)^2}{4} + (k+1)^3 \\
    &= \frac{k^2(k+1)^2}{4} + \frac{4(k+1)^3}{4} \\
    &= \frac{k^2(k+1)^2 + 4(k+1)^3}{4} \\
    &= \frac{(k+1)^2(4(k+1) + k^2)}{4} \\
    &= \frac{(k+1)^2(k^2 + 4k + 4)}{4} \\
    &= \frac{(k+1)^2(k + 2)^2}{4}
\end{align*}
$\therefore P(k) \then P(k + 1) \qed$
\end{enumerate}
\item Prove that for every positive integer $n$, $3^{2n} - 1$ is divisible by 8.
\begin{enumerate}
\item Basis step:
\begin{align*}
    P(1): &\quad 8 \mid 3^{2 \cdot 1} - 1 \iff 8 \mid 3^2 - 1 \iff 8 \mid 8 \\
    P(2): &\quad 8 \mid 3^{2 \cdot 2} - 1 \iff 8 \mid 3^4 - 1 \iff 8 \mid 80
\end{align*}
$\therefore P(1), P(2)$
\item Induction step: \\
Assume $P(k)$.
\[8 \mid 3^{2k} - 1 \iff \exists q \in \Z \left(\frac{3^{2k} - 1}{8} = q\right)\]
\begin{align*}
    q &= \frac{3^{2k} - 1}{8} \\
    8q &= 3^{2k} - 1 \\
    8 \cdot 3^{2k} + 8q &= 3^{2k} - 1 + 8 \cdot 3^{2k} \\
    8(q + 3^{2k}) &= 9(3^{2k}) - 1 \\
    8(q + 3^{2k}) &= 3^2 \cdot 3^{2k} - 1 \\
    8(q + 3^{2k}) &= 3^{2k + 2} - 1 \\
    8(q + 3^{2k}) &= 3^{2(k + 1)} - 1 \\
    q + 3^{2k} &= \frac{3^{2(k + 1)} - 1}{8}
\end{align*}
\begin{gather*}
    k \in \N \wedge q \in \Z \implies q + 3^{2k} \in \Z \\
    \implies 8 \mid 3^{2(k+1)} - 1
\end{gather*}
$\therefore P(k) \then P(k + 1) \qed$
\pagebreak
\end{enumerate}
\item Prove that $\dfrac{n^3}{3} + \dfrac{n^5}{5} + \dfrac{7n}{15}$ is an integer for every 
positive integer $n$.
\begin{enumerate}
\item Basis step:
\begin{align*}
    &P(1): \frac{1^3}{3} + \frac{1^5}{5} + \frac{7}{15} \in \Z \\
    &\iff \frac{1}{3} + \frac{1}{5} + \frac{7}{15} \in \Z \\
    &\iff \frac{5}{15} + \frac{3}{15} + \frac{7}{15} \in \Z \\
    &\iff \frac{15}{15} \in \Z \\
    &\iff 1 \in \Z
\end{align*}
$\therefore P(1)$
\item Induction step: \\
Assume $P(k)$.
\begin{align*}
    & \frac{k^3}{3} + \frac{k^5}{5} + \frac{7k}{15} \in \Z \\
    \iff & \frac{5k^3}{15} + \frac{3k^5}{15} + \frac{7k}{15} \in \Z \\
    \iff & \frac{3k^5 + 5k^3 + 7k}{15} \in \Z \\
    \iff & 15 \mid 3k^5 + 5k^3 + 7k
\end{align*}
We want $P(k + 1)$.
\begin{align*}
    P(k + 1): & \frac{(k+1)^3}{3} + \frac{(k+1)^5}{5} + \frac{7(k+1)}{15} \in \Z \\
    \iff & 15 \mid 3(k+1)^5 + 5(k+1)^3 + 7(k+1) \in \Z \\
\end{align*}
\begin{align*}
    & 3(k+1)^5 + 5(k+1)^3 + 7(k+1) \\
    &= (3k^5 + 15k^4 +30k^3 +30k^2 + 15k + 3) + (5k^3 + 15k^2 + 15k + 5) + (7k + 7) \\
    &= 3k^5 + 15k^4 + 35k^3 + 45k^2 + 37k + 15 \\
    &= (3k^5 + 5k^3 + 7k) + 15k^4 + 30k^3 + 45k^2 + 30k + 15 \\
    &= (3k^5 + 5k^3 + 7k) + 15(k^4 + 2k^3 + 3k^2 + 2k + 1)
\end{align*}
\begin{gather*}
    k^4 + 2k^3 + 3k^2 + 2k + 1 \in \Z \implies 15 \mid 15(k^4 + 2k^3 + 3k^2 + 2k + 1) \\
    P(k): 15 \mid 3k^5 + 5k^3 + 7k \equiv T \\
    \implies 15 \mid (3k^5 + 5k^3 + 7k) + 15(k^4 + 2k^3 + 3k^2 + 2k + 1) \\
    \therefore 15 \mid 3(k+1)^5 + 5(k+1)^3 + 7(k+1)
\end{gather*}
$\therefore P(k) \then P(k+1) \qed$
\end{enumerate}
\item Assume that a chocolate bar consists of $n$ squares arranged in a rectangular pattern. 
The entire bar, or any smaller rectangular piece of the bar, can be broken along a horizontal 
or vertical line separating the squares. Assuming that only one piece can be broken at a time, 
prove that $n-1$ successive breaks are needed to separate the bar into its $n$ separate squares, 
using strong induction.
\begin{enumerate}
    \item Basis step
    \begin{flushleft}
        A whole chocolate bar (1 piece) when divided (1 break) leaves 2 pieces.
        \[m = 2 \text{ pieces from } m - 1 = 2 - 1 = 1 \text{ pieces}\]
    \end{flushleft}
    $\therefore P(2)$
    \item Induction step
    \begin{flushleft}
        Assume $P(j) \quad 1 \le j \le n$. Upon the first break of the bar, two rectangles are 
        left, each composed of $a$ and $b$ squares, respectively.
        \[a + b = n \qquad a < n, \quad b < n\]
        Since we assumed $P(j)$, the breaks needed to separate the first part into its $a$ squares 
        is $a-1$ and similarly $b-1$ breaks for the second's $b$ squares. But we also must take 
        into account the break we did in order to separate $a$ and $b$.
        \[1 + (a-1) + (b-1) = (a + b) - 1= n - 1 \qed\]
        Therefore, $n - 1$ breaks are necessary to separate a bar into its $n$ squares.
    \end{flushleft}
\end{enumerate}
\end{enumerate}
\pagebreak
Let $f_n$ be the $n$th Fibonacci number, defined by $f_1 = 1$, $f_2 = 1$, and 
$f_n = f_{n-1} + f_{n-2}$ for all $n \ge 3$.
\begin{enumerate}[resume]
\setcounter{enumi}{4}
\item Show that $f_{n+6} = 4f_{n+3} + f_n$ for all positive integers $n$.
\begin{align*}
    f_{n+6} &= f_{n+5} + f_{n+4} \\
    &= f_{n+4} + f_{n+3} + f_{n+4} \\
    &= f_{n+3} + f_{n+2} + f_{n+3} + f_{n+3} + f_{n+2} \\
    &= 3f_{n+3} + f_{n+2} + f_{n+2} \\
    &= 3f_{n+3} + f_{n+2} + f_{n+1} + f_n \\
    &= 4f_{n+3} + f_n
\end{align*}
$\therefore f_{n+6} = 4f_{n+3} + f_n \qed$
\item Make and prove a conjecture about sums of the form $f_2 + f_4 + f_6 + \dots + f_{2n}$.
\[f = (1, 1, 2, 3, 5, 8, 13, 21, 34, \dots)\]
\begin{align*}
    f_2 + f_4 &= 1 + 3 &= \;\:4 &= \;\,5 - 1 &= f_5 - 1 \\
    f_2 + f_4 + f_6 &= 1 + 3 + 8 &= 12 &= 13 - 1 &= f_7 - 1 \\
    f_2 + f_4 + f_6 + f_8 &= 1 + 3 + 8 + 21 &= 33 &= 34 - 1 &= f_9 - 1
\end{align*}
\[\sum_{i = 1}^{n} f_{2i} = f_2 + f_4 + f_6 + \dots + f_{2n} = f_{2n + 1} - 1\]
Assume $P(k)$
\begin{align*}
    \sum_{i = 1}^{k} f_{2i} &= f_{2k + 1} - 1 \\
    f_{2k + 2} + \sum_{i = 1}^{k} f_{2i} &= f_{2k + 1} - 1 + f_{2k + 2} \\
    f_2 + f_4 + f_6 + \dots + f_{2k} + f_{2k + 2} &= (f_{2k + 2} + f_{2k + 1}) - 1 \\
    \sum_{i = 1}^{k+1} f_{2i} &= f_{2k + 3} - 1 \\
    \sum_{i = 1}^{k+1} f_{2i} &= f_{2(k + 1)} - 1
\end{align*}
$\therefore P(k) \then P(k + 1) \qed$
\end{enumerate}
\end{document}
