\documentclass[a4paper, 12pt]{article}
\usepackage{amsmath, amssymb}

\newtheorem{theorem}{Theorem}[section]
\newtheorem{lemma}[theorem]{Lemma}
\newtheorem{proposition}[theorem]{Proposition}
\newtheorem{corollary}[theorem]{Corollary}

\newenvironment{proof}[1][Proof]{\begin{trivlist}
\item[\hskip \labelsep {\bfseries #1}]}{\end{trivlist}}
\newenvironment{definition}[1][Definition]{\begin{trivlist}
\item[\hskip \labelsep {\bfseries #1}]}{\end{trivlist}}
\newenvironment{example}[1][Example]{\begin{trivlist}
\item[\hskip \labelsep {\bfseries #1}]}{\end{trivlist}}
\newenvironment{remark}[1][Remark]{\begin{trivlist}
\item[\hskip \labelsep {\bfseries #1}]}{\end{trivlist}}

\newcommand{\keyword}[1]{\textbf{#1}}

\title{MATH 381 HW 1}
\author{Christian Jahnel}
\date{24 January 2024}

\begin{document}
    \maketitle
    \begin{enumerate}
    \item \hfill
        \begin{enumerate}
        \item $p \rightarrow q$
        \item $\neg p \wedge \neg q \wedge r$
        \item $q \wedge \neg p$
        \end{enumerate}
    \item \hfill
        \begin{enumerate}
        \item There are zeros in the decimal expansion of $\pi$.
        \item The number 3 is not positive or the number 4 is not odd (or both).
        \end{enumerate}
    \item \hfill
        \begin{itemize}
        \item hypothesis: if you access the website
        \item conclusion: you have to register your device or be within the temporary access period
        \item converse: If you have to register your device or be within the temporary access period,
        then you access the website.
        \item inverse: If you do not access the website,
        then you do not have to register your device nor be within the temporary access period.
        \item contrapositive: If you do not have to register your device nor be within the temporary access period,
        then you do not access the website.
        \item negation: You access the website and you do not have to register your device nor be within the temporary access period.
        \end{itemize}
    \item Given that $x \rightarrow y$ is false
        \begin{enumerate}
        \item
            \[x \wedge y\]
            $x \rightarrow y$ is only false when $x$ is true and $y$ is false.
            Since both $x$ and $y$ would need to be true for the conjunction of $x$ and $y$
            to be true, it is \keyword{false}.
        \item
            \[(\neg x) \rightarrow y \equiv \neg(\neg x) \vee y \equiv x \vee y\]
            Since this conditional statement can be reduced to the disjunction of $x$ and $y$,
            and we know that $x$ must be true and $y$ must be false for $x \rightarrow y$
            to be false, only one condition is needed to fulfill the disjunction,
            and the $x$ is true, so this is \keyword{true}.
        \item
            \[x \vee \neg y\]
            Since we know that $x$ must be true and $y$ must be false for $x \rightarrow y$
            to be false, both conditions for this disjunction are fulfilled, so it is \keyword{true}.
        \end{enumerate}
    \item \hfill
    \end{enumerate}
\end{document}