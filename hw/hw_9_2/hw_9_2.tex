\documentclass[letterpaper, 12pt]{article}
\usepackage{amsmath, amssymb}

\newtheorem{theorem}{Theorem}[section]
\newtheorem{lemma}[theorem]{Lemma}
\newtheorem{proposition}[theorem]{Proposition}
\newtheorem{corollary}[theorem]{Corollary}

\newenvironment{proof}[1][Proof]{\begin{trivlist}
\item[\hskip \labelsep {\bfseries #1}]}{\end{trivlist}}
\newenvironment{definition}[1][Definition]{\begin{trivlist}
\item[\hskip \labelsep {\bfseries #1}]}{\end{trivlist}}
\newenvironment{example}[1][Example]{\begin{trivlist}
\item[\hskip \labelsep {\bfseries #1}]}{\end{trivlist}}
\newenvironment{remark}[1][Remark]{\begin{trivlist}
\item[\hskip \labelsep {\bfseries #1}]}{\end{trivlist}}

\newcommand{\qed}{\quad \blacksquare}
\newcommand{\keyword}[1]{\textbf{#1}}
\newcommand{\then}{\rightarrow}
\newcommand{\bithen}{\leftrightarrow}

\newcommand{\N}{\mathbb{N}}
\newcommand{\Z}{\mathbb{Z}}
\newcommand{\Q}{\mathbb{Q}}
\newcommand{\R}{\mathbb{R}}
\newcommand{\C}{\mathbb{C}}
\newcommand{\0}{\emptyset}

\title{MATH 381 HW 9 part 2}
\author{Christian Jahnel}
\date{27 March 2024}

\begin{document}
\maketitle
\begin{enumerate}
\item Find all integer solutions to the following. If there are no integer solutions, explain why.
\begin{enumerate}
\item $3 + 2x \equiv -2 \pmod 7$
\begin{align*}
    3 + 2x &\equiv -2 \pmod 7 \\
    2x &\equiv -5 \pmod 7 \\
    2x + 0 &\equiv -5 + 7 \pmod 7 \\
    2x &\equiv 2 \pmod 7 \\
    x &\equiv 1 \pmod 7 \\
    x \in \hat{1} &= \{z \in \Z \mid z \bmod 7 = 1\} \\
    &= \{\dots, -13, -6, 1, 8, 15, \dots\}
\end{align*}
\item $2x - 4 \equiv 0 \pmod 6$
\begin{align*}
    2x - 4 &\equiv 0 \pmod 6 \\
    2x &\equiv 4 \pmod 6 \\
    x &\equiv 2 \pmod 6 \\
    x \in \hat{2} &= \{z \in \Z \mid z \bmod 6 = 2\} \\
    &= \{\dots, -10, -4, 2, 8, 14, \dots\}
\end{align*}
\item $x + y \equiv x - y \pmod 5$
\begin{align*}
    x + y &\equiv x - y \pmod 5 \\
    y &\equiv -y \pmod 5 \\
    1 &\equiv -1 \pmod 5 \qquad y \ne 0\\
    1 &\equiv 4 \pmod 5
\end{align*}
Let $y = 0$.
\begin{align*}
    x + 0 &\equiv x - 0 \pmod 5 \\
    x &\equiv x \pmod 5 \\
    x &\in \Z
\end{align*}
\begin{flushleft}
    The equation can be simplified to the equivalent equation that 1 and -1 are equivalent 
    modulo 5. This is a contradiction because 1 and -1 are in their own equivalence classes: 
    $\hat{1}$ and $\hat{4}$, respectively. However, this simplification assume $y \ne 0$. 
    Consequently, in the case where $y = 0$, the equation is satisfied for any $x \in \Z$. 
    Therefore, there are inifinite pairs of solutions: $\{(x, 0) \mid x \in \Z\}$.
\end{flushleft}
\end{enumerate}
\pagebreak
\item Prove that for all integers $n \ge 0$, $10^n \equiv 1 \pmod 9$. Then, use that result to show 
that a positive integer is divisible by 9 if and only if the sum of its digits is divisible by 9. \\
Basis step
\begin{align*}
    10^0 &\equiv 1 \pmod 9 \\
    \iff 1 &\equiv 1 \pmod 9 \\
    \therefore P(0)
\end{align*}
Inductive step; assume $P(k)$.
\begin{align*}
    10^k &\equiv 1 \pmod 9 \\
    10^k \cdot 10 &\equiv 1 \cdot 1 \pmod 9 \\
    10^{k+1} &\equiv 1 \pmod 9 \\
    \therefore P(k) \then P(k+1)
\end{align*}
$\therefore \forall n \in \{z \in \Z \mid z \ge 0\} (10^n \equiv 1 \pmod 9) \qed$ \\
\\
Every positive integer can be written as a sum of its digits weighted by its place value in base-10.
\[k = k_0 + 10k_1 + 100k_2 + \dots + 10^nk_n = \sum_{i=0}^{n} 10^ik_i\]
\begin{align*}
    10^0 \equiv 1 \pmod 9 &\implies 10^0k_0 \equiv k_0 \mod 9 \\
    10^1 \equiv 1 \pmod 9 &\implies 10^1k_1 \equiv k_1 \mod 9 \\
    10^2 \equiv 1 \pmod 9 &\implies 10^2k_2 \equiv k_2 \mod 9 \\
    &\vdots \\
    10^n \equiv 1 \pmod 9 &\implies 10^nk_n \equiv k_n \mod 9
\end{align*}
$\therefore \forall 0 \le i \le n (10^ik^i \equiv k_i \pmod 9)$
\begin{align*}
    9 \mid k \iff 0 &\equiv k \pmod 9 \\
    0 &\equiv \sum_{i=0}^{n} 10^ik_i \pmod 9 \\
    0 &\equiv \sum_{i=0}^{n} k_i \pmod 9
\end{align*}
$\therefore 9 \mid k \iff 9 \mid (k_0 + k_1 + k_2 + \dots + k_n) \qed$
\item Show that if $n$ is any integer, then $n^2$ is congruent modulo 4 to either 0 or 1.
\[n \in \Z \then n^2 \equiv 0 \vee n^2 \equiv 1 \pmod 4\]
If $n$ is odd:
\[\exists k \in \Z (n = 2k + 1)\]
\begin{align*}
    n^2 &= (2k + 1)^2 \\
    &= 4k^2 + 4k + 1 \\
    &= 4(k^2 + k) + 1 \\
    \implies n^2 &\equiv 4(k^2 + k) + 1 \pmod 4 \\
    &\equiv 1 \pmod 4
\end{align*}
If $n$ is even:
\[\exists k \in \Z (n = 2k)\]
\begin{align*}
    n^2 &= (2k)^2 \\
    &= 4k^2 \\
    \implies n^2 &\equiv 4(k^2) \pmod 4 \\
    &\equiv 0 \pmod 4
\end{align*}
Therefore, $n^2$ is congruent to either 0 or 1 modulo 4 for all integers, since all integers 
must be either odd or even. $\qed$
\end{enumerate}
\end{document}
