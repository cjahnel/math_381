\documentclass[letterpaper, 12pt]{article}
\usepackage{amsmath, amssymb}

\newtheorem{theorem}{Theorem}[section]
\newtheorem{lemma}[theorem]{Lemma}
\newtheorem{proposition}[theorem]{Proposition}
\newtheorem{corollary}[theorem]{Corollary}

\newenvironment{proof}[1][Proof]{\begin{trivlist}
\item[\hskip \labelsep {\bfseries #1}]}{\end{trivlist}}
\newenvironment{definition}[1][Definition]{\begin{trivlist}
\item[\hskip \labelsep {\bfseries #1}]}{\end{trivlist}}
\newenvironment{example}[1][Example]{\begin{trivlist}
\item[\hskip \labelsep {\bfseries #1}]}{\end{trivlist}}
\newenvironment{remark}[1][Remark]{\begin{trivlist}
\item[\hskip \labelsep {\bfseries #1}]}{\end{trivlist}}

\newcommand{\qed}{\nobreak \ifvmode \relax \else
    \ifdim\lastskip<1.5em \hskip-\lastskip
    \hskip1.5em plus0em minus0.5em \fi \nobreak
    \vrule height0.75em width0.5em depth0.25em\fi}
\newcommand{\keyword}[1]{\textbf{#1}}
\newcommand{\then}{\rightarrow}
\newcommand{\bithen}{\leftrightarrow}

\title{MATH 381 Homework 3}
\author{Olivia Dumitrescu}
\date{17 February 2024}

\begin{document}
\maketitle
\begin{enumerate}
\item Show that $(p \then q) \vee (p \then r)$ is logically equivalent to $p \then (q \vee r)$.
\begin{align*}
    &\quad \; (p \then q) \vee (p \then r) & \\
    &\equiv \neg p \vee q \vee \neg p \vee r \qquad
    &[\text{Conditional-disjunction equivalence}]\\
    &\equiv \neg p \vee \neg p \vee q \vee r \qquad
    &[\text{Commutative law}]\\
    &\equiv \neg p \vee (q \vee r) \qquad
    &\text{[Idempotent law]} \\
    &\equiv p \then (q \vee r) \qquad
    &\text{[Conditional-disjunction equivalence]}
\end{align*}
\item Show that $(q \wedge (p \then \neg q)) \then \neg p$ is a tautology using propositional 
equivlance and the laws of logic (successive substitution). Do not use a truth table.
\begin{align*}
    &\quad\;& (q \wedge (p \then \neg q)) &\then \neg p \\
    &\equiv & \neg(q \wedge (p \then \neg q)) &\vee \neg p \qquad
    &\text{[Conditional-disjunction equivalence]} \\
    &\equiv & (\neg q \vee \neg(p \then \neg q)) &\vee \neg p \qquad
    &\text{[DeMorgan's law]} \\
    &\equiv & (\neg q \vee \neg(\neg p \vee \neg q)) &\vee \neg p \qquad
    &\text{[Conditional-disjunction equivalence]} \\
    &\equiv & (\neg q \vee (p \wedge q)) &\vee \neg p \qquad
    &\text{[DeMorgan's law]} \\
    &\equiv & ((\neg q \vee p) \wedge (\neg q \vee q)) &\vee \neg p \qquad
    &\text{[Distributive law]} \\
    &\equiv & ((\neg q \vee p) \wedge T) &\vee \neg p \qquad
    &\text{[Negation law]} \\
    &\equiv & (\neg q \vee p) &\vee \neg p \qquad
    &\text{[Identity law]} \\
    &\equiv & \neg q \vee (p &\vee \neg p) \qquad
    &\text{[Associative law]} \\
    &\equiv & \neg q &\vee T \qquad
    &\text{[Negation law]} \\
    &\equiv & &T \qquad
    &\text{[Domination law]}
\end{align*}
\pagebreak
\item Show that $(p \wedge q) \then r$ and $(p \then r) \wedge (q \then r)$ are not logically equivalent.
\begin{center}
    \begin{tabular}{c | c | c | c | c | c | c | c}
        $p$ & $q$ & $r$ & $(p \wedge q)$ & $(p \wedge q) \then r$ & $(p \then r)$ & $(q \then r)$ 
        & $(p \then r) \wedge (q \then r)$ \\
        \hline
        T & T & T & T & T & T & T & T\\
        T & T & F & T & F & F & F & F\\
        T & F & T & F & T & T & T & T\\
        T & F & F & F & *T & F & T & *F\\
        F & T & T & F & T & T & T & T\\
        F & T & F & F & *T & T & F & *F\\
        F & F & T & F & T & T & T & T\\
        F & F & F & F & T & T & T & T\\
    \end{tabular}
\end{center}
\begin{flushleft}
    For the cases in which $p$ is true and $q$ and $r$ are false, and $q$ is true and $p$ and 
    $r$ are false, $(p \wedge q) \then r$ and $(p \then r) \wedge (q \then r)$ have different 
    truth values. Therefore, they are not logically equivalent.
\end{flushleft}
\item Create a compound logical expression composed of at least two propositions ($p$, $q$, e.g.) 
that is a contradiction. Show that your expression really is a contradiction.
\begin{align*}
    &\quad \; (\neg q \wedge \neg p) \wedge (p \vee q) \\
    &\equiv (\neg q \wedge (p \vee q)) \wedge (\neg p \wedge (p \vee q)) \qquad
    &\text{[Distributive law]} \\
    &\equiv ((\neg q \wedge p) \vee (\neg q \wedge q)) \wedge ((\neg p \wedge p) \vee (\neg p \wedge q)) \qquad
    &\text{[Distributive law]} \\
    &\equiv ((\neg q \wedge p) \vee F) \wedge (F \vee (\neg p \wedge q)) \qquad
    &\text{[Negation law]} \\
    &\equiv (\neg q \wedge p) \wedge (\neg p \wedge q) \qquad
    &\text{[Identity law]} \\
    &\equiv \neg q \wedge (p \wedge \neg p) \wedge q \qquad
    &\text{[Associative law]} \\
    &\equiv \neg q \wedge F \wedge q \qquad
    &\text{[Negation law]} \\
    &\equiv F \qquad
    &\text{[Domination law]}
\end{align*}
\pagebreak
\item Determine whether each argument is valid or invalid. Explain your reasoning.
\begin{enumerate}
\item \parbox{3in}{$\begin{aligned}
    p &\then q \\
    p &\then r \\
    \hline
    \therefore q &\then r
\end{aligned}$}
\begin{align*}
    &\quad & ((p \then q) \wedge (p \then r)) \then (q \then r) \\
    &\equiv & \neg((p \then q) \wedge (p \then r)) \vee (q \then r) & 
    &\parbox{1in}{[Conditional \\-disjunction \\ equivalence]} \\
    &\equiv & (\neg(p \then q) \vee \neg(p \then r)) \vee (q \then r) & 
    &\text{[DeMorgan's law]} \\
    &\equiv & \neg(\neg p \vee q) \vee \neg(\neg p \vee r) \vee (\neg q \vee r) & 
    &\parbox{1in}{[Conditional \\-disjunction \\ equivalence]} \\
    &\equiv & (p \wedge \neg q) \vee (p \wedge \neg r) \vee \neg q \vee r & 
    &\text{[DeMorgan's law]} \\
    &\equiv & \neg q \vee (p \wedge \neg q) \vee r \vee (p \wedge \neg r) & 
    &\text{[Commutative law]} \\
    &\equiv & \neg q \vee r \vee (p \wedge \neg r) & 
    &\text{[Absorption law]} \\
    &\equiv & \neg q \vee ((r \vee p) \wedge (r \vee \neg r)) & 
    &\text{[Distributive law]} \\
    &\equiv & \neg q \vee ((r \vee p) \wedge T) & 
    &\text{[Negation law]} \\
    &\equiv & \neg q \vee r \vee p & 
    &\text{[Identity law]}
\end{align*}
\begin{flushleft}
    This is an \textbf{invalid} argument because the premises and conclusion can be strung 
    together to form a conditional statement which is not a tautology, as it has been fully 
    simplified to reveal a contingency. \\
    Intuitively, $q$ and $r$ are only dependent on $p$, which is not even stated to be 
    necessarily true. Since there is no direct relationship between $q$ and $r$, one can 
    conclude that the argument is not valid.
\end{flushleft}
\item \parbox{3in}{$\begin{aligned}
    (p \then q) &\wedge (r \then s) \\
    p &\vee r \\
    \hline
    \therefore q &\vee s
\end{aligned}$}
\begin{align*}
    &\quad & (((p \then q) \wedge (r \then s)) \wedge (p \vee r)) \then (q \vee s) & \\
    &\equiv & \neg(((p \then q) \wedge (r \then s)) \wedge (p \vee r)) \vee q \vee s & 
    &\parbox{1in}{[Conditional \\-disjunction \\ equivalence]} \\
    &\equiv & \neg((p \then q) \wedge (r \then s)) \vee \neg(p \vee r) \vee q \vee s & 
    &\text{[DeMorgan's law]} \\
    &\equiv & \neg(p \then q) \vee \neg(r \then s) \vee \neg(p \vee r) \vee q \vee s & 
    &\text{[DeMorgan's law]} \\
    &\equiv & \neg(\neg p \vee q) \vee \neg(\neg r \vee s) \vee \neg(p \vee r) \vee q \vee s & 
    &\parbox{1in}{[Conditional \\-disjunction \\ equivalence]} \\
    &\equiv & (p \wedge \neg q) \vee (r \wedge \neg s) \vee (\neg p \wedge \neg r) \vee q \vee s & 
    &\text{[DeMorgan's law]} \\
    &\equiv & q \vee (p \wedge \neg q) \vee s \vee (r \wedge \neg s) \vee (\neg p \wedge \neg r) & 
    &\text{[Commutative law]} \\
    &\equiv & ((q \vee p) \wedge (q \vee \neg q)) 
    \vee ((s \vee r) \wedge (s \vee \neg s)) 
    \vee (\neg p \wedge \neg r) & 
    &\text{[Distributive law]} \\
    &\equiv & ((q \vee p) \wedge T) 
    \vee ((s \vee r) \wedge T) 
    \vee (\neg p \wedge \neg r) & 
    &\text{[Negation law]} \\
    &\equiv & (q \vee p) \vee (s \vee r) \vee (\neg p \wedge \neg r) & 
    &\text{[Identity law]} \\
    &\equiv & q \vee s \vee r \vee (p \vee (\neg p \wedge \neg r)) & 
    &\text{[Commutative law]} \\
    &\equiv & q \vee s \vee r \vee ((p \vee \neg p) \wedge (p \vee \neg r)) & 
    &\text{[Distributive law]} \\
    &\equiv & q \vee s \vee r \vee (T \wedge (p \vee \neg r)) & 
    &\text{[Negation law]} \\
    &\equiv & q \vee s \vee r \vee p \vee \neg r & 
    &\text{[Identity law]} \\
    &\equiv & q \vee s \vee p \vee (r \vee \neg r) & 
    &\text{[Commutative law]} \\
    &\equiv & q \vee s \vee p \vee T & 
    &\text{[Negation law]} \\
    &\equiv & T & 
    &\text{[Domination law]}
\end{align*}
\begin{flushleft}
    This is a \textbf{valid} argument because the premises and conclusion can be strung together 
    to form a conditional statement which can be simplified to a tautology. \\
    Intuitively, it is suggesting that two conditional statements are true and one of the 
    hypotheses are true, so it is logical that one of the conclusions is true, i.e. the one 
    corresponding with the hypothesis that fulfills the disjunction.
\end{flushleft}
\end{enumerate}
\pagebreak
\item Determine whether each of the following propositions is true or false, with explanation.
\begin{enumerate}
\item $\forall x P(x) \then \exists x P(x)$, where $P(x)$ is an arbitrary predicate function.
\begin{flushleft}
    This is necessarily \textbf{true} because if $P(x)$ holds for every $x$ in the domain, then 
    there certainly exists an $x$ for which $P(x)$ holds. Only a single solution satisfies the
    conclusion, and the hypothesis states that every single input satisfies the conclusion.
\end{flushleft}
\item $\exists x P(x) \then \forall x P(x)$, where $P(x)$ is an arbitrary predicate function.
\begin{flushleft}
    This is not necessarily true and therefore \textbf{false} in the general case because the 
    hypothesis only supposes that at least one input exists such that $P(x)$ holds, so it is 
    not certain that every element satisfies the conclusion. While it is true that it is 
    \textit{possible} that every input satisfies the conclusion, the hypothesis only gurantees 
    one, and therefore the conclusion cannot be proved from the hypothesis alone.
\end{flushleft}
\item $\forall x(2x \ge x)$, where the domain consists of all real numbers.
\[2x \ge x \implies
\begin{cases}
    2 \ge 1 & x > 0 \\
    2(0) \ge 0 & x = 0 \\
    2 \le 1 & x < 0
\end{cases}\]
\begin{flushleft}
    The predicate $2x \ge x$ is equivalent to $2 \ge 1$ (which is obviously true) by the division
    property of equality for $x \in \mathbb{R}, \; x > 0$. For the case that $x = 0$,
    $2(0) \ge 0 \implies 0 \ge 0$, which is obviously true. For the case that $x < 0$, the 
    predicate is equivalent to the statement $2 \le 1$ (since the inequality must flip under 
    division by a negative number). This is obviously false. Therefore, the universal quanitfier 
    is \textbf{false} since not every case for $x$ makes the predicate hold.
\end{flushleft}
\pagebreak
\item $\exists x(e^x = x^2)$, where the domain consists of all real numbers.
\begin{align*}
    e^x &= x^2 \\
    \ln(e^x) &= \ln(x^2) \\
    x &= 2\ln(x)
\end{align*}
\begin{align*}
    f(x) &= x & \frac{df}{dx} &= 1 & \frac{d^2f}{dx^2} &= 0 \\
    g(x) &= 2\ln(x) & \frac{dg}{dx} &= \frac{2}{x} & \frac{d^2g}{dx^2} &= \frac{-2}{x^2}
\end{align*}
\begin{align*}
    f(x) &> 0 &\quad g(x) &< 0 &\quad g(x) &< f(x) &\quad 0 < &x < 1 \\
    f(x) &= 1 &\quad g(x) &= 0 &\quad g(x) &< f(x) &\quad x &= 1 \\
    f(x) &= e^{\frac{1}{2}} &\quad g(x) &= 1 & g(x) &< f(x) &\quad x &= e^{\frac{1}{2}} \\
    f(x) &= 2 &\quad g(x) &= \ln(4) &\quad g(x) &< f(x) &\quad x &= 2 \\
    \dfrac{df}{dx} &= 1 &\quad \dfrac{dg}{dx} &< 1 
    &\quad \dfrac{dg}{dx} &< \dfrac{df}{dx} 
    &\quad x &> 2 \\
    \dfrac{d^2f}{dx^2} &= 0 &\quad \dfrac{d^2g}{dg^2} &< 0 
    &\quad \dfrac{d^2g}{dg^2} &< \dfrac{d^2f}{dx^2} 
    &\quad x &\in \mathbb{R} \setminus \{0\}
\end{align*}
\begin{flushleft}
    The predicate produces the equivalent equation $x = 2\ln(x)$ when taking the natural 
    logarithm of both sides. If we declare two functions $f$ and $g$ to represent each side 
    of the equation, we can analyze them directly. Since $f$ and $g$ are both continuous, 
    monotonic functions, we can compare their values and derivatives at certain intervals to 
    see if there is a possible solution to $f(x) = g(x)$. For $x \in (0, 2]$, $g(x) < f(x)$, 
    so there are no solutions where $x \le 2$ (and we've ruled out $x \le 0$ because $\ln(x)$ 
    is only defined for $x \in \mathbb{R}_+$). For $x > 2$, there could be a possible solution 
    if the graphs of $f$ and $g$ intersect. However, the slope of $g$ is less steep than that 
    of $f$ for $x > 2$. Furthermore, the slope of $g$ only decreases as $x$ increases, whereas 
    the slope of $f$ remains the same for all $x$. Consequently, the graph of $f$ has a headstart 
    over that of $g$ at $x = 2$, and the graph of $g$ cannot ``catch up'' to the graph of $f$. 
    Therefore, the existential quanitfier is \textbf{false}, as there is no $x \in \mathbb{R}$ 
    for which the predicate holds.
\end{flushleft}
\end{enumerate}
\end{enumerate}
\end{document}