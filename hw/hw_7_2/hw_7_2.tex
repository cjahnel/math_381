\documentclass[letterpaper, 12pt]{article}
\usepackage{amsmath, amssymb}
\usepackage{enumerate}

\newtheorem{theorem}{Theorem}[section]
\newtheorem{lemma}[theorem]{Lemma}
\newtheorem{proposition}[theorem]{Proposition}
\newtheorem{corollary}[theorem]{Corollary}

\newenvironment{proof}[1][Proof]{\begin{trivlist}
\item[\hskip \labelsep {\bfseries #1}]}{\end{trivlist}}
\newenvironment{definition}[1][Definition]{\begin{trivlist}
\item[\hskip \labelsep {\bfseries #1}]}{\end{trivlist}}
\newenvironment{example}[1][Example]{\begin{trivlist}
\item[\hskip \labelsep {\bfseries #1}]}{\end{trivlist}}
\newenvironment{remark}[1][Remark]{\begin{trivlist}
\item[\hskip \labelsep {\bfseries #1}]}{\end{trivlist}}

\newcommand{\qed}{\quad \blacksquare}
\newcommand{\keyword}[1]{\textbf{#1}}
\newcommand{\then}{\rightarrow}
\newcommand{\bithen}{\leftrightarrow}

\newcommand{\N}{\mathbb{N}}
\newcommand{\Z}{\mathbb{Z}}
\newcommand{\Q}{\mathbb{Q}}
\newcommand{\R}{\mathbb{R}}
\newcommand{\C}{\mathbb{C}}
\newcommand{\0}{\emptyset}

\title{MATH 381 HW 7 part 2}
\author{Christian Jahnel}
\date{10 March 2024}

\begin{document}
\maketitle
\begin{enumerate}
\item Prove that the following are equivalent for any subsets $A$ and $B$ of the same universal 
set $U$:
\begin{enumerate}
\item $A \subseteq B$;
\item $A \cap \bar{B} = \0$;
\item $\bar{A} \cup B = U$.
\end{enumerate}
\begin{enumerate}[i.]
    \item $A \subseteq B \implies A \cup \bar{B} = \0$
    \begin{align*}
        A \subseteq B &\equiv \forall x \in U(x \in A \then x \in B) \\
        &\equiv \forall x \in U(\neg(x \in A) \vee x \in B) \\
        &\equiv \forall x \in U(x \notin A \vee x \in B) \\
        &\equiv \forall x \in U(x \in \bar{A} \vee x \in B) \\
        &\equiv \forall x \in U(x \in \bar{A} \cup B) \\
        &\implies \bar{A} \cup B = U
    \end{align*}
    $\therefore a \then c$
    \item $\bar{A} \cup B = U \implies A \cap \bar{B} = \0$
    \begin{align*}
        \bar{A} \cup B &= U \\
        \implies \overline{\bar{A} \cup B} &= \bar{U} \\
        \implies \overline{\bar{A}} \cap \bar{B} &= \bar{U} \\
        \implies A \cap \bar{B} &= \bar{U} \\
        &= \{x \in U \mid x \notin U\} \\
        &= \0
    \end{align*}
    $\therefore c \then b$
    \item $A \cap \bar{B} = \0 \implies A \subseteq B$
    \begin{align*}
        A \cap \bar{B} &= \0 \\
        \implies A - B &= \0 \\
        \implies \{x \mid x \in A \wedge x \notin B\} &= \0
    \end{align*}
    \begin{align*}
        \{x \mid x \in A \wedge x \notin B\} = \0 
        &\implies \nexists x \in U(x \in A \wedge x \notin B) \\
        &\equiv \neg(\exists x \in U (x \in A \wedge x \notin B)) \\
        &\equiv \forall x \in U (\neg(x \in A) \vee x \in B) \\
        &\equiv \forall x \in U (x \in A \then x \in B) \\
        &\implies A \subseteq B
    \end{align*}
    $\therefore b \then a \qed$
\end{enumerate}
\item Prove or disprove: for any sets $A$, $B$, and $C$, if $A \cup B = B \cup C$, then $A = C$.
Let $A = \{1\}, B = \{1, 2\}, C = \{2\}$.
\[
\begin{aligned}
    A \cup B &= \{1, 2\} = B \\
    B \cup C &= \{1, 2\} = B
\end{aligned}
\implies A \cup B = B \cup C
\]
$\therefore A \cup B = B \cup C \wedge A \ne C \qed$
\begin{center}
\begin{tabular}{c | c | c | c | c}
    $A$ & $B$ & $C$ & $A \cup B$ & $B \cup C$ \\
    \hline
    1 & 1 & 1 & 1 & 1 \\
    *1* & 1 & *0* & *1* & *1* \\
    1 & 0 & 1 & 1 & 1 \\
    1 & 0 & 0 & 1 & 0 \\
    *0* & 1 & *1* & *1* & *1* \\
    0 & 1 & 0 & 1 & 1 \\
    0 & 0 & 1 & 0 & 1 \\
    0 & 0 & 0 & 0 & 0 \\
\end{tabular}
\end{center}
\begin{flushleft}
    As seen in the above table, there are two cases in which $A \cup B = B \cup C$ does not 
    imply that $A = C$. The starred rows show such cases. Since $B$ already contains the 
    element that happens to be in $A$ and not $C$ or vice-versa depending on the case, the 
    duplicate is not counted, since its membership in $B$ makes it qualify anyway. Therefore, 
    \textbf{the proposition has been disproven generally}.
\end{flushleft}
\item A U.S. Social Security number consists of 9 digits; the first digit may be a 0.
\begin{enumerate}
\item How many different Social Security numbers are there?
\[
    10^9 = 1,000,000,000
\]
\item How many Social Security numbers have all even digits?
\[
    5^9 = 1,953,125
\]
\item Determine the minimum population required to gurantee that at least 15 people have the 
same Social Security number. \\
\[
    14 \cdot 1,000,000,000 + 1 = 14,000,000,001
\]
(By the Pigeonhole Principle)
\end{enumerate}
\item A group of 12 women and 12 men are to be lined up.
\begin{enumerate}
\item How many lines are there beginning with a woman and ending with a man?
\begin{flushleft}
    The line would look like this: \\
    W \_ \_ \_ \_ \_ \_ \_ \_ \_ \_ \_ \_ \_ \_ \_ \_ \_ \_ \_ \_ \_ \_ M \\
    Since we have 12 options for the woman at the beginning and 12 options for the man at the end, 
    by the product rule these numbers multiply with the number of permutations of the remaining 
    22 people i.e. $22!$.
\end{flushleft}
\[
    12 \cdot 22! \cdot 12 \approx 1.61856105\times10^{23}
\]
\item How many lines are available if no two men are allowed to be adjacent?
\begin{flushleft}
    Since no two men are allowed to be adjacent, the line must alternate between a man and a 
    woman. Therefore, we can reduce the problem to two situations depending on the gender of 
    the first in line.
    \begin{enumerate}
        \item MWMWMWMWMWMWMWMWMWMWMWMW
        \item WMWMWMWMWMWMWMWMWMWMWMWM
    \end{enumerate}
    If we then only think about the order of the women and men separately, we can reduce the 
    problem to the number of arrangements of men multiplied by the number of arrangements of 
    women (by the product rule). We also have to multiply this by two since we consider two 
    possibilities based on the gender of the person first in line.
\end{flushleft}
\begin{align*}
    P_M &= 12! \\
    P_W &= 12! \\
    P &= 2 \cdot (12! \cdot 12!) \\
    &\approx 4.58885066\times10^{17}
\end{align*}
\end{enumerate}
\item Let $S$ be the set of all integers that are not divisible by 17, and let $T$ be any subset 
of $S$ with $|T|=308$. Show that there must be at least twenty integers in $T$ that have the 
same remainder when divided by 17.
\begin{gather*}
    a = dq + r \iff r = a \bmod d \qquad r \in \Z_m \\
    \qquad |\Z_m| = 17 \\
    \forall a \in T (a \nmid 17) \implies \forall a \in T (r = a \bmod 17 \ne 0) \implies 
    \forall a \in T (r \in \Z_m \setminus \{0\}) \\
    |\Z_m \setminus \{0\}| = 17 - 1 = 16
    \intertext{Therefore, there are 16 possible remainders for every element in the set.}
    \intertext{We must place 308 objects into 16 boxes.}
    \lceil \tfrac{308}{16} \rceil = 20
    \intertext{By the Pigeonhole Principle, there must exist at least one box containing at 
    least 20 objects.}
    \intertext{Therefore, this means there must exist at least 20 integers in the set $T$ which have the 
    same remainder when diivided by 17. $\qed$}
\end{gather*}
\end{enumerate}
\end{document}
