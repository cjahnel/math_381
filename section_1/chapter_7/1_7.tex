\documentclass[a4paper, 12pt]{article}
\usepackage{amsmath, amssymb}

\newtheorem{theorem}{Theorem}[section]
\newtheorem{lemma}[theorem]{Lemma}
\newtheorem{proposition}[theorem]{Proposition}
\newtheorem{corollary}[theorem]{Corollary}

\newenvironment{proof}[1][Proof]{\begin{trivlist}
\item[\hskip \labelsep {\bfseries #1}]}{\end{trivlist}}
\newenvironment{definition}[1][Definition]{\begin{trivlist}
\item[\hskip \labelsep {\bfseries #1}]}{\end{trivlist}}
\newenvironment{example}[1][Example]{\begin{trivlist}
\item[\hskip \labelsep {\bfseries #1}]}{\end{trivlist}}
\newenvironment{remark}[1][Remark]{\begin{trivlist}
\item[\hskip \labelsep {\bfseries #1}]}{\end{trivlist}}

\newcommand{\qed}{\nobreak \ifvmode \relax \else
    \ifdim\lastskip<1.5em \hskip-\lastskip
    \hskip1.5em plus0em minus0.5em \fi \nobreak
    \vrule height0.75em width0.5em depth0.25em\fi}
\newcommand{\keyword}[1]{\textbf{#1}}
\newcommand{\then}{\rightarrow}
\newcommand{\bithen}{\leftrightarrow}

\title{MATH 381 Section 1.7}
\author{Olivia Dumitrescu}
\date{31 January 2024}

\begin{document}
    \maketitle
    \section*{Introduction to Proofs}
    \begin{definition}
        \hfill
        \begin{enumerate}
            \item The integer $n =$ even number if $\exists$ an integer $k$ so that $n = 2k$.
            \item The integer $n =$ odd number if $\exists$ an integer $k$ so that $n = 2k + 1$.
        \end{enumerate}
    \end{definition}
    If $x \in \mathbb{Z}$, either $x =$ odd or $x = $ even.
    \begin{example}
        Give a proof for the statement \\
        "If $n =$ odd integer, then $n^2$ is an odd integer". \\
        \begin{align*}
            \text{proof } P(n): n &= \text{ odd integer} \\
            Q(n): n^2 &= \text{ odd integer}
        \end{align*}
        claim $\forall n (P(n) \then Q(n))$
    \end{example}
    \begin{example}
        Give a direct proof to \\
        "If $m$ and $n$ are both perfect squares, then $m \times n$ is also a perfect square."
        \begin{proof}
            Since $m$ and $n$ are both perfect squares $\implies \exists a, b \in \mathbb{Z}^+$
            so that
            \begin{enumerate}
                \item $m \times n = a^2 \times b^2 = (ab)^2$ \qed
            \end{enumerate}
        \end{proof}
    \end{example}
    \subsection*{Section 1.7.6 proof by negation}
    Assume we want to prove a conditional statement $p \then q \equiv \neg q \then \neg p$.
    \begin{enumerate}
        \item prove it by \keyword{contraposition} i.e. prove
        \[\neg q \then \neg p.\]
        \item prove it by \keyword{contradiction} i.e. assume \\
        $p \then q$ is False i.e. $p = T$ and $q = F$.
    \end{enumerate}
    \begin{example} %4 because 3 is missing and I took a picture
        Prove that if $n = a \cdot b$ where $a$ and $b$ are positive integers,
        then $a \le \sqrt{n}$ or $b \le \sqrt{n}$.
        \begin{align*}
            q: \quad a \le \sqrt{n} \vee b \le \sqrt{n} \\
            \neg q: \quad \neg(a \le \sqrt{n}) &\wedge \neg(b \le \sqrt{n}) \\
            (a > \sqrt{n}) &\wedge (b > \sqrt{n})
        \end{align*}
    \end{example}
    \begin{example} %5
        Prove that $\sqrt{2}$ is an irrational number.
        \begin{proof}
            Assume by contradiction that $\sqrt{2} \in \mathbb{Q}$.
            \[\mathbb{Q} = \{\frac{a}{b}, \; (a, b) = 1, \; b \ne 0; \; a, b \in \mathbb{Z}\}\]
            \[\sqrt{2} \in \mathbb{Q} \implies \exists a, b \in \mathbb{Z}, \; b \ne 0, \; (a, b) = 1\]
            \begin{align*}
                \sqrt{2} &= \frac{a}{b} \\
                \implies 2 &= (\frac{a}{b})^2 = \frac{a^2}{b^2} \\.
                \implies a^2 &= 2b^2
            \end{align*}
        \end{proof}
    \end{example}
    \subsection*{Proof of Equivalences}
    To prove a theorem that is a biconditional statement of the form $p \bithen q$,
    you must prove two conditional statements: $p \then q$ and $q \then p$.
    \[p \bithen q = (p \then q) \wedge (q \then p)\]
    Sometimes a theorem stating that many propositions are equivalent
    \[p_1, p_2, p_3, \dots, p_n\]
    Show that TFAE (that the following are equivalent)
    \begin{enumerate}
        \item $p_1$
        \item $p_2$
        \item $p_3$
        \item[n.] $p_n$
    \end{enumerate}
    bu definition it means prove
    \begin{align*}
        p_1 &\bithen p_2 \\
        p_2 &\bithen p_3 \\
        &\; \; \vdots \\
        p_{n-1} &\bithen p_n
    \end{align*}
    i.e. $2(n - 1)$ statements to prove
    \begin{theorem}
        If we can prove a loop in the form
        \begin{align*}
            p_1 &\then p_2 \\
            p_2 &\then p_3 \\
            &\; \; \vdots \\
            p_{n-1} &\then p_n \\
            p_{n} &\then p_1
        \end{align*}
        i.e. $n$ statements then it suffices to show that $p_1, \dots, p_n$
        are equivalent statements.
    \end{theorem}
    \begin{example}
        Let $n \in \mathbb{Z}$ \\
        Prove that $n$ is odd if and only if $n^2$ is odd.
        \begin{proof}
            \begin{align*}
                p: \text{$n$ is odd} \\
                q: \text{$n^2$ is odd}
            \end{align*}
        \end{proof}
        \begin{enumerate}
            \item Show that $p \then q$ i.e. if $n$ is odd then $n^2$ is odd.
            \item Show that $q \then p$ i.e. if $n^2$ is odd then $n$ is odd.
            We prove $q \then p$ by showing the contrapositive $\neg p \then \neg q$
            i.e. if $n$ is even then $n^2$ is even.
        \end{enumerate}
    \end{example}
    \begin{example}
        Let $n \in \mathbb{Z}$. \\
        Show that the following are equivalent:
        \begin{align*}
            p_1: \text{$n$ is even} \\
            p_2: \text{$n - 1$ is odd} \\
            p_3: \text{$n^2$ is even}
        \end{align*}
        $p_1 \then p_2$ is obvious. \\
        $p_2 \then p_3$ can be proven with another integer $k$. \\
        $p_3 \then p_1$ is proven by its equivalence with $\neg p_1 \then \neg p_3$.
    \end{example}
    \begin{example}
        Show that the statement ``every positive integer is the sum of two squares of integers''
        is false. \\ Counterexample is $n = 3$.
    \end{example}
\end{document}