\documentclass[a4paper, 12pt]{article}
\usepackage{amsmath, amssymb}

\newtheorem{theorem}{Theorem}[section]
\newtheorem{lemma}[theorem]{Lemma}
\newtheorem{proposition}[theorem]{Proposition}
\newtheorem{corollary}[theorem]{Corollary}

\newenvironment{proof}[1][Proof]{\begin{trivlist}
\item[\hskip \labelsep {\bfseries #1}]}{\end{trivlist}}
\newenvironment{definition}[1][Definition]{\begin{trivlist}
\item[\hskip \labelsep {\bfseries #1}]}{\end{trivlist}}
\newenvironment{example}[1][Example]{\begin{trivlist}
\item[\hskip \labelsep {\bfseries #1}]}{\end{trivlist}}
\newenvironment{remark}[1][Remark]{\begin{trivlist}
\item[\hskip \labelsep {\bfseries #1}]}{\end{trivlist}}

\newcommand{\keyword}[1]{\textbf{#1}}

\title{MATH 381 Section 1.1}
\author{Olivia Dumitrescu}
\date{10 January 2024}

\begin{document}
    \maketitle
    \section{Get the title of this section}
    \subsection{Introduction}
    \begin{definition}
        A \keyword{proposition} is a sentence that declares a fact that
        is either true or false but not both.
    \end{definition}
    \begin{definition}
        If $p$ is a proposition, then $\bar{p}$ or $\neg p$ is the \keyword{negation}
        of proposition $p$, i.e. ``it is not the case that $p$ [holds]''.
    \end{definition}
    \begin{definition}
        Assume that $p$ and $q$ are propositions. Then the
        \keyword{conjucntion} of $p$ and $q$, $p \wedge q$, is the
        proposition $p$ \keyword{and} $q$.
    \end{definition}
    \begin{remark}
        $p \wedge q$ is true when both $p$ \keyword{and} $q$
        are true. It is false otherwise.
    \end{remark}
    \begin{definition}
        Let $p$ and $q$ be two propositions. We define the
        \keyword{disjunction} of $p$ and $q$, $p \vee q$ to be
        proposition $p$ \keyword{or} $q$.
    \end{definition}
    \begin{remark}
        $p \vee q$ is false when both $p$ and $q$ are false.
        It is true otherwise.
    \end{remark}
    \begin{remark}
        If we have $n$ propositions, its associated table contains $2^n$ rows.
    \end{remark}
    \begin{definition}
        $p$ and $q$ are two propositions.\\
        The \keyword{exclusive} of $p$ and $q$, $p \oplus q$, is to be a proposition
        that is true exactly when either $p$ or $q$ is true but not both.
        \[ p \oplus q = p \vee q - p \wedge q \]
    \end{definition}
    \begin{example}
        \hfill \break
        $p$: ``A student can have a salad with dinner.''\\
        $q$: ``A student can have soup with dinner.''\\
        Express
        \begin{itemize}
            \item $p \wedge q$ ``A student can have both a salad
            and soup with dinner.''
            \item $p \vee q$ ``A studet can have a salad or soup or
            both.''
            \item $p \oplus q$ ``A student can have dinner with
            either salid or soup.''
        \end{itemize}
    \end{example}
    \begin{definition}
        Let $p$ and $q$ be two propositions.\\
        The \keyword{conditional statement} $p \rightarrow q$ is a
        proposition: ``if $p$ then $q$''.
    \end{definition}
    \begin{remark}
        $p \rightarrow q$ is false if $p =$ true and $q =$ false and
        true otherwise.
    \end{remark}
    \begin{remark}
        Terminology:
        \begin{itemize}
            \item if $p$ then $q$
            \item if $p$, $q$
            \item $p$ is sufficient for $q$
            \item $q$ if $p$
            \item $q$ when $p$
            \item a necessary condition for $q$ is $p$
            \item $q$ unless $\neg p$
            \item $q$ provided that $p$
        \end{itemize}
    \end{remark}
    \begin{proof}
        Prove that the exclusive $p \rightarrow q$ is logicaly equivalent
        to the contrapositive $\neg q \rightarrow \neg p$.\\
        (i.e. the two exlusive propositions have the same T or F values)
    \end{proof}
    \begin{definition}
        The \keyword{biconditional statement} $p \leftrightarrow q$
        is the proposition ``$p$ if and only if $q$''.
        It is true whenever $p$ and $q$ have the same value.
    \end{definition}
    \begin{proof}
        Is the biconditional statement $p \leftrightarrow q$ logically
        equivalent to $(p \rightarrow q) \wedge (q \rightarrow p)$?
        Yes $\because$ the implication between $p$ and $q$ goes both ways
        $\therefore p \leftrightarrow q \iff (p \rightarrow q) \wedge
        (q \rightarrow p)$.
    \end{proof}
    \begin{definition}
        \hfill
        \begin{enumerate}
            \item The \keyword{converse} of the conditional statement
            $p \rightarrow q$ is $q \rightarrow p$.
            \item The \keyword{contrapositive} of the conditional statement
            $p \rightarrow q$ is $\neg q \rightarrow \neg p$.
            \item The \keyword{inverse} of the conditional statement
            $p \rightarrow q$ is $\neg p \rightarrow \neg q$.
        \end{enumerate}
    \end{definition}
    \begin{definition}
        \hfill
        \begin{enumerate}
            \item A \keyword{bit} is a symbol $\{0, 1\}$.
            \item A \keyword{boolean variable} is a variable that can
            either be true or false.
            \item The \keyword{length} of a bitstring is the number
            of bits in the string.
        \end{enumerate}
    \end{definition}
    \begin{example}
        bitwise\\
        $x: = 01 1011 0110$\\
        $y: = 11 0001 1101$\\
        $x \vee y: = 11 1011 1111$\\
        $x \wedge y: = 01 0001 0100$\\
        $x \oplus y: = 10 1010 1011$\\
    \end{example}
\end{document}