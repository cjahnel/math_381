\documentclass[a4paper, 12pt]{article}
\usepackage{amsmath, amssymb}

\newtheorem{theorem}{Theorem}[section]
\newtheorem{lemma}[theorem]{Lemma}
\newtheorem{proposition}[theorem]{Proposition}
\newtheorem{corollary}[theorem]{Corollary}

\newenvironment{proof}[1][Proof]{\begin{trivlist}
\item[\hskip \labelsep {\bfseries #1}]}{\end{trivlist}}
\newenvironment{definition}[1][Definition]{\begin{trivlist}
\item[\hskip \labelsep {\bfseries #1}]}{\end{trivlist}}
\newenvironment{example}[1][Example]{\begin{trivlist}
\item[\hskip \labelsep {\bfseries #1}]}{\end{trivlist}}
\newenvironment{remark}[1][Remark]{\begin{trivlist}
\item[\hskip \labelsep {\bfseries #1}]}{\end{trivlist}}

\newcommand{\keyword}[1]{\textbf{#1}}

\title{MATH 381 Section 1.5}
\author{Olivia Dumitrescu}
\date{24 January 2024}

\begin{document}
    \maketitle
    \section*{Nested Quantifiers}
    \[\forall x \exists y Q(x, y)\]
    $x$ and $y$ are free variables within $Q(x, y)$
    \begin{example}
        Let $Q(x, y): x + y = 0; x, y \in \mathbb{R}$\\
        What are the truth values for the quantifiers?
        \begin{align*}
            \exists y \forall x Q(x, y) &= \text{true: }
            \text{There exists a $y$ so that regardless of $x$, we have $y + x = 0$.}\\
            \forall x \exists y Q(x, y) &= \text{true: }
            \text{For any $x$ there exists a $y$ so that $x + y = 0$.}
        \end{align*}
    \end{example}
    \begin{example}
        Let $Q(x, y, z)$ be $x + y = z$; $x, y, z \in \mathbb{R}$.\\
        Find truth values
        \begin{enumerate}
            \item $\forall x \forall y \exists z Q(x, y, z)$ is True
            \item $\exists z \forall x \forall y Q(x, y, z)$ is False
        \end{enumerate}
    \end{example}
    \subsection*{The negation of nested qualifiers}
    \begin{enumerate}
        \item \[\neg(\forall x \forall y \exists z Q(x, y, z)) = F\]
        \[\equiv \exists x \exists y \forall z (\neg Q(x, y, z))\]
        \[\equiv \exists x \exists y \forall z (x + y \ne z)\]
        \item \[\exists z \forall x \forall y Q(x, y, z) = F\]
        \[\neg(\exists z \forall x \forall y Q(x, y, z)) \equiv T\]
        \[\equiv \forall z \exists x \exists y \neg Q(x, y, z)\]
    \end{enumerate}
    Take $f: \mathbb{R} \rightarrow \mathbb{R}$ and $A, L \in \mathbb{R}$.
    \[\lim_{x \rightarrow A} f(x) = L\]
    \begin{definition}
        For any $\epsilon > 0$ (consider $\epsilon$ to be small) there exists a $\delta > 0$
        so that if $|x - A| < \delta$ then $|f(x) - L| < \epsilon$.
        [delta becomes a function of epsilon]
    \end{definition}
    \begin{example}
        Express the definition of a limit of a real-valued function as quantifiers.
        Then, negate it.
        \begin{enumerate}
            \item The definition of $\lim_{x \rightarrow A} f(x) = L$ is
            \[\forall \epsilon > 0 \; \exists \delta_\epsilon > 0 \text{ so that }
            (|x - A| < \delta \rightarrow |f(x) - L| < \epsilon)\]
            \item Write $\lim_{x \rightarrow A} f(x)$ does not exist
            \begin{enumerate}
                \item $\lim_{x \rightarrow A} f(x)$ exists?
                \[\exists L \; \forall \epsilon > 0 \; \exists \delta_\epsilon > 0 \;
                (|x - A| < \delta \rightarrow |f(x) - L| < \epsilon)\]
                \item $\lim_{x \rightarrow A} f(x)$ does not exist if
                \[\neg(\exists L \; \forall \epsilon > 0 \; \exists \delta_\epsilon > 0 \;
                (|x - A| < \delta \rightarrow |f(x) - L| < \epsilon))\]
                \[\equiv \forall L \; \exists \epsilon > 0 \; \forall \delta_\epsilon > 0 \;
                \neg(|x - A| < \delta \rightarrow |f(x) - L| < \epsilon)\]
                \[\equiv \forall L \; \exists \epsilon > 0 \; \forall \delta_\epsilon > 0 \;
                (|x - A| < \delta \wedge |f(x) - L| \ge \epsilon)\]
            \end{enumerate}
        \end{enumerate}
    \end{example}
    \begin{example}
        Mean Value Theorem
    \end{example}
    Any Theorem is a conditional statement.
    \[p \rightarrow q\]
    $p$ is the hypothesis, $q$ is the conclusion.\\
    To prove a Theorem is false i.e. find a counterexample i.e. find an f(x) within
    hypothesis where the conclusion does not hold.
    \begin{example}
        Every non-zero real number has a multiplicative inverse.
        \[\forall x \; \exists y \; (x \ne 0 \implies xy = 1); \quad
        x, y \in \mathbb{R}\]
        \begin{definition}
            \hfill
            \begin{enumerate}
                \item A ring is a set with an addition operation where the additive inverse
                exists for every member and is itself a member.
                \item A field is a set with a multiplication operation where the
                multiplicative inverse exists for every member and is itself a member..
            \end{enumerate}
        \end{definition}
        \begin{itemize}
            \item $(\mathbb{R}, +, dot)$ is a ring and a field
            \item $(\mathbb{Q}, +, dot)$ is a ring and a field
            \item $(\mathbb{C}, +, dot)$ is a ring and a field
            \item $(\mathbb{Z}, +, dot)$ is a ring
        \end{itemize}
    \end{example}
\end{document}