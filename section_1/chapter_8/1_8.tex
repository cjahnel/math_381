\documentclass[letterpaper, 12pt]{article}
\usepackage{amsmath, amssymb}

\newtheorem{theorem}{Theorem}[section]
\newtheorem{lemma}[theorem]{Lemma}
\newtheorem{proposition}[theorem]{Proposition}
\newtheorem{corollary}[theorem]{Corollary}

\newenvironment{proof}[1][Proof]{\begin{trivlist}
\item[\hskip \labelsep {\bfseries #1}]}{\end{trivlist}}
\newenvironment{definition}[1][Definition]{\begin{trivlist}
\item[\hskip \labelsep {\bfseries #1}]}{\end{trivlist}}
\newenvironment{example}[1][Example]{\begin{trivlist}
\item[\hskip \labelsep {\bfseries #1}]}{\end{trivlist}}
\newenvironment{remark}[1][Remark]{\begin{trivlist}
\item[\hskip \labelsep {\bfseries #1}]}{\end{trivlist}}

\newcommand{\qed}{\nobreak \ifvmode \relax \else
    \ifdim\lastskip<1.5em \hskip-\lastskip
    \hskip1.5em plus0em minus0.5em \fi \nobreak
    \vrule height0.75em width0.5em depth0.25em\fi}
\newcommand{\keyword}[1]{\textbf{#1}}
\newcommand{\then}{\rightarrow}
\newcommand{\bithen}{\leftrightarrow}

\title{MATH 381 Section 1.8}
\author{Prof. Olivia Dumitrescu}
\date{5 February 2024}

\begin{document}
    \maketitle
    \section*{Proof methods and strategies}
    Suppose we want to prove
    \[p_1 \vee p_2 \vee p_3 \vee \dots \vee p_n \then q\]
    is equivalent to
    \[(p \then q) \wedge (p_2 \then q) \wedge \dots \wedge (p_n \then q)\].
    \begin{example}
        Prove $(n + 1)^3 \ge 3^n$ for positive integers $n \le 4$.
    \end{example}
    \begin{example}
        Prove that for any $n \in \mathbb{Z}, \: n^2 \ge n$
    \end{example}
    \begin{example}
        Use a proof by cases to show that
        \[|x \cdot y| = |x| \cdot |y| \quad x, y \in \mathbb{R}\]
    \end{example}
    \begin{example}
        Show there are no integer solutions $x$ and $y$ to the following
        \[x^2 + 3y^2 = 18\]
    \end{example}
    \subsection*{Fermat's Last Theorem}
    The equation $x^n + y^n = z^n$ has no integer solutions for $n > 2$.
    \begin{example}
        Show that if $x$ and $y$ are even numbers, then both $xy$ and $x + y$ are even integers.
    \end{example}
    \begin{example}
        Open question: The $3x + 1$ or Collatz conjecture \\
        \[\forall x \in \mathbb{Z}, \: \exists n \; (T^n(x) = 1)\]
        Let $T$ be the transformation:
        \begin{enumerate}
            \item if $x$ is an even integer $\then \frac{x}{2}$
            \item if $x$ is an odd interger $\then T(x) = 3x + 1$
        \end{enumerate}
    \end{example}
    \begin{example}
        Two distinct positive numbers $x$ and $y$
        \[\frac{x + y}{2} > \sqrt{xy}\]
        i.e. the arithmetic mean is greater than the geometric mean
    \end{example}
    \begin{proof}
        Recall that for positive numbers
        \[x^2 > y^2 \iff x > y\]
        which can be proven through difference of squares.
        \[\frac{x + y}{2} \ge \sqrt{xy}\]
        It suffices to prove
        \[(\frac{x + y}{2})^2 \ge (\sqrt{xy})^2\]
    \end{proof}
    \subsection*{Uniqueness Proofs}
    $P(x)$: desired property
    \begin{enumerate}
        \item $\exists x (P(x) \wedge \forall y ( y \ne x \then \neg P(y)))$
        \item Assume both $x$ and $y$ satisfy $P(x)$ show that $x = y$.
    \end{enumerate}
    \begin{example}
        Show that every line that is not horizontal has a unique solution.
    \end{example}
    \begin{proof}
        \hfill
        \begin{enumerate}
            \item existence
            \item uniqueness
        \end{enumerate}
        Assume $x_1$ and $x_2$ are 2 solutions for a line. Claim: $x_1 = x_2$
        \[y = ax + b\]
        is not horizontal if $a \ne 0$ \\
    \end{proof}
    \begin{example}
        Prove that there exists two irrational numbers $x, y$ so that $x^y \in \mathbb{Q}$.
    \end{example}
    \begin{proof}
        \[\exists x, y \notin \mathbb{Q} \text{ s.t. } x^y \in \mathbb{Q}\]
    \end{proof}
\end{document}