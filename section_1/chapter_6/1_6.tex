\documentclass[a4paper, 12pt]{article}
\usepackage{amsmath, amssymb}

\newtheorem{theorem}{Theorem}[section]
\newtheorem{lemma}[theorem]{Lemma}
\newtheorem{proposition}[theorem]{Proposition}
\newtheorem{corollary}[theorem]{Corollary}

\newenvironment{proof}[1][Proof]{\begin{trivlist}
\item[\hskip \labelsep {\bfseries #1}]}{\end{trivlist}}
\newenvironment{definition}[1][Definition]{\begin{trivlist}
\item[\hskip \labelsep {\bfseries #1}]}{\end{trivlist}}
\newenvironment{example}[1][Example]{\begin{trivlist}
\item[\hskip \labelsep {\bfseries #1}]}{\end{trivlist}}
\newenvironment{remark}[1][Remark]{\begin{trivlist}
\item[\hskip \labelsep {\bfseries #1}]}{\end{trivlist}}

\newcommand{\keyword}[1]{\textbf{#1}}
\newcommand{\then}{\rightarrow}

\title{MATH 381 Section 1.6}
\author{Olivia Dumitrescu}
\date{26 January 2024}

\begin{document}
    \maketitle
    \section*{Rules of Inference}
    \begin{definition}
        An argument in propositional logic is a sequence of propositions where all but
        the final proposition are premises and the final one is the conclusion.
    \end{definition}
    \begin{remark}
        An argument is valid if the truth values of all its premises implies the truth
        of its conclusion.
    \end{remark}
    \begin{example}
        \hfill
        \begin{enumerate}
            \item Determine whether the argument is valid
            \item Determine if the conclusion is true
        \end{enumerate}
        If $\sqrt{2} > \frac{3}{2}$, then $(\sqrt{2})^2 > \frac{9}{4}$.
        \begin{enumerate}
            \item true $\because p \then q$
            \item false $\because q$ is false
        \end{enumerate}
    \end{example}
    \begin{center}
    \begin{tabular}{c c | c | c}
        & Rules of Inference & Tautology & Name \\
        \hline
        1 & \parbox{2cm}{\begin{gather*}
            p \\
            p \then q \\
            \hline
            \therefore q
        \end{gather*}}
        & $p \wedge (p \then q) \then q$
        & Modus ponens \\
        \hline
        2 & \parbox{2cm}{\begin{gather*}
            \neg q \\
            p \then q \\
            \hline
            \therefore \neg p
        \end{gather*}}
        & $\neg q \wedge (p \then q) \then \neg p$
        & Modus tollens \\
        \hline
        3 & \parbox{2cm}{\begin{gather*}
            p \then q \\
            q \then r \\
            \hline
            \therefore p \then r
        \end{gather*}}
        & $(p \then q) \wedge (q \then r) \then (p \then r)$
        & Hypothetical syllogism \\
        \hline
        4 & \parbox{2cm}{\begin{gather*}
            p \vee q \\
            \neg p \\
            \hline
            \therefore q
        \end{gather*}}
        & $(p \vee q) \wedge \neg p \then q$
        & Disjunctive Syllogism \\
        \hline
        5 & \parbox{2cm}{\begin{gather*}
            p \\
            \hline
            \therefore p \vee q
        \end{gather*}}
        & $p \then p \vee q$
        & Addition \\
        \hline
        6 & \parbox{2cm}{\begin{gather*}
            p \wedge q \\
            \hline
            \therefore p
        \end{gather*}}
        & $p \wedge q \then p$
        & Simplification \\
        \hline
        7 & \parbox{2cm}{\begin{gather*}
            p \vee q \\
            \neg p \vee r \\
            \hline
            \therefore q \vee r
        \end{gather*}}
        & $(p \vee q) \wedge (\neg p \vee r) \then q \vee r$
        & Resolution \\
    \end{tabular}
    \end{center}
    \begin{remark}
        The propostiion $(p \then q) \wedge q \then p$ is not a tautology i.e. it has a false
        value if $p = F$ and $q = T$. This is because it is not birectional implication. \\
        Similarly, the proposition $(p \then q) \wedge \neg p \then \neg q$ is not a tautology.
        This is because if $p$ is false, then $p \then q$ will always be true no matter $q$.
        It is akin to taking the contrapositive and its conclusion to assert the hypothesis.
    \end{remark}
    \begin{example}
        If it rains today, then we will not have a barbecue today. \\
        If we don't have a barbecue today, then we will have a barbecue tomorrow. \\
        Therefore, if it rains today, then we will have a barbecue tomorrow. \\
        $p$: It rains today \\
        $q$: We will not have a barbecue today \\
        $r$: We will have a barbecue tomorrow \\
        \begin{center}
            \parbox{3cm}{\begin{gather*}
                p \then q \\
                q \then r \\
                \hline
                \therefore p \then r
            \end{gather*}}
            \quad Hypothetical Syllogism
        \end{center}
    \end{example}
    \begin{example}
        Premises
        \begin{enumerate}
            \item If you send me an email message, then I will finish writing the program. \\
            $p \then q$
            \item If you do not send me an email message, then I will go to sleep early. \\
            $\neg p \then r$
            \item If I go to sleep early, then I will wake up feeling refreshed. \\
            $r \then s$
        \end{enumerate}
        Conclusion \\
        If I do not finish writing the program, then I will wake up feeling refreshed.\\
        $\neg q \then s$\\
        \newline
        propositions \\
        $p$: You send me an email message \\
        $q$: I will finish writing the program \\
        $r$: I will go to sleep early \\
        $s$: I will wake up feeling refreshed \\
        \begin{align}
            p \then q \iff \neg q &\then \neg p \\
            \neg p &\then r \\
            %hline
            \neg q &\then r \quad \quad [Hypothetical Syllogism] \\
            r &\then s \\
            %hline
            \neg q &\then s \quad \quad [Hypothetical Syllogism]
        \end{align}
    \end{example}
    \begin{example}
        hypothesis
        \begin{enumerate}
            \item Jasmine is skiing or it is not snowing. \\
            $p \vee \neg q$
            \item It is snowing or Bant is playing hockey. \\
            $q \vee r$
            \item Implies Jasmine is skiing or Bant is playing hockey \\
            $p \vee r$
        \end{enumerate}
        \begin{enumerate}
            \item $p$: Jasmine is skiing.
            \item $q$: It is snowing.
            \item $r$: Bant is playing hockey.
        \end{enumerate}
        \begin{align*}
            p &\vee \neg q \\
            q &\vee r \\
            %hline
            p &\vee r \quad \quad [Hypothetical Syllogism]
        \end{align*}
    \end{example}
    \begin{example}
        Show that premises imply the conclusion $p \vee s$
        \begin{enumerate}
            \item \begin{align*}
                (p \wedge q) \vee r = (p \vee r) \wedge (q \vee r) \\
            \end{align*}
            \item \begin{align*}
                r \then s = \neg r \vee s
            \end{align*}
            \item \begin{align*}
                (p &\vee r) \\
                \neg r &\vee s \\
                %hline
                p &\vee s \quad \quad \text{Resolution}
            \end{align*}
            \item \begin{align*}
                p &\wedge q \\
                %hline
                &p \quad \quad \text{Simplification}
            \end{align*}
        \end{enumerate}
    \end{example}
    \subsection*{Fallacies}
    Is the proposition a tautology?
    \[(p \then q) \wedge q \then p\]
    \begin{example}
        Is the following argument valid?
        \begin{enumerate}
            \item If you solve every problem in the textbook, then you will learn discrete
            mathematics. \\
            $p \then q$
            \item You did learn discrete mathematics. \\
            $q$
        \end{enumerate}
        Therefore, you solved every problem in the book. \\
        This is a fallacy, because $q$ does not imply $p$ by $p \then q$
    \end{example}
    \begin{example}
        Premises
        \begin{enumerate}
            \item Everyone in Discrete Math has taken a course in Computer Science.
            \item Manta is a student in this class.
            \item \keyword{Conclusion}: Manta has taken a course in computer science.
        \end{enumerate}
        Domain is UNC Students. \\
        \begin{itemize}
            \item $A(x)$: $x$ is a student in this class.
            \item $B(x)$: $x$ has taken a Computer Science course.
        \end{itemize}
        \begin{enumerate}
            \item $\forall x (A(x) \then B(x))$
            \item Manta $\in A(x)$.
            \item \keyword{Conclusion}: Manta has taken a course in Computer Science.
        \end{enumerate}
    \end{example}
\end{document}