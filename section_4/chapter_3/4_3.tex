\documentclass[letterpaper, 12pt]{article}
\usepackage{amsmath, amssymb}
\usepackage{mathtools}

\newtheorem{theorem}{Theorem}[section]
\newtheorem{lemma}[theorem]{Lemma}
\newtheorem{proposition}[theorem]{Proposition}
\newtheorem{corollary}[theorem]{Corollary}

\newenvironment{proof}[1][Proof]{\begin{trivlist}
\item[\hskip \labelsep {\bfseries #1}]}{\end{trivlist}}
\newenvironment{definition}[1][Definition]{\begin{trivlist}
\item[\hskip \labelsep {\bfseries #1}]}{\end{trivlist}}
\newenvironment{example}[1][Example]{\begin{trivlist}
\item[\hskip \labelsep {\bfseries #1}]}{\end{trivlist}}
\newenvironment{remark}[1][Remark]{\begin{trivlist}
\item[\hskip \labelsep {\bfseries #1}]}{\end{trivlist}}

\newcommand{\qed}{\quad \blacksquare}
\newcommand{\keyword}[1]{\textbf{#1}}
\newcommand{\then}{\rightarrow}
\newcommand{\bithen}{\leftrightarrow}
\DeclareMathOperator{\lcm}{lcm}

\newcommand{\N}{\mathbb{N}}
\newcommand{\Z}{\mathbb{Z}}
\newcommand{\Q}{\mathbb{Q}}
\newcommand{\R}{\mathbb{R}}
\newcommand{\C}{\mathbb{C}}
\newcommand{\0}{\emptyset}
\newcommand{\power}{\mathcal{P}}

\newcommand{\divby}{\; \vdots \;}

\title{MATH 381 Section 4.3}
\author{Prof. Olivia Dumitrescu}
\date{22 March 2024}

\begin{document}
    \maketitle
    \section*{Section 4.3 Primes and GCDs}
    Recall for any $n \in \N$, $1 \mid n$, $n \mid n$.
    \begin{definition}
        Let $p$ be an integer where $p \ge 2$. \\
        $p$ is prime if its only positive factors are 1 and $p$. \\
        An integer greater than 1 is called composite if it's not prime. \\
        Note: 1 is neither prime nor composite.
    \end{definition}
    \begin{theorem}
        Fundamental Theorem of Arithmetic \\
        Every integer greater than 1 is either prime or uniquely a product of primes.
    \end{theorem}
    \begin{theorem}
        If $n$ is a composite integer, then it has a prime factor less than or equal to $\sqrt{n}$.
    \end{theorem}
    \begin{proof}
        $n$ is composite
        \[
            \begin{aligned}
                1 < a < n \\
                n = a \cdot b
            \end{aligned}
            \implies 1 < b < n
        \]
        Claim is false if $a > \sqrt{n} \wedge b > \sqrt{n}$.
        \[a \cdot b > \sqrt{n} \cdot \sqrt{n} = n \implies n > n\]
        Therefore, claim is true by contradiction.
    \end{proof}
    \begin{example}
        Find prime factorization of 7007.
    \end{example}
    \begin{theorem}
        There are infiinitely many primes.
    \end{theorem}
    \begin{proof}
        Assume by contradiction this is not the case and we have finitely many primes.
        \[\{p_1, \dots, p_n\}\]
        \[q := p_1 \cdot p_2 \cdot p_3 \cdot \dots \cdot p_n + 1\]
        \begin{enumerate}
            \item has a unique prime factorization
            \item or is a prime number
        \end{enumerate}
        \begin{gather*}
            \exists 1 \le j \le n (p_j \mid q) \\
            p_j \mid p_1 \cdot p_2 \cdot \dots \cdot p_n + 1 \implies p_j \mid 1
        \end{gather*}
        i.e. there is no prime number dividing $q$. Therefore $q$ has to be prime $q > p_n$ 
        so it is not on the list. Therefore, there are infinitely many primes by contradiction.
    \end{proof}
    \begin{theorem}
        The ratio of $\pi(x)$ (the number of primes, not exceeding $x$), and $\dfrac{x}{\ln x}$ 
        approaches 1 when $x$ gets larger and larger.
    \end{theorem}
    \begin{proposition}
        Goldbach's conjecture \\
        1742 Goldbach wrote to Euler: Every odd integer is a sum of 3 primes $n > 5$ \\
        Euler simplified it: Every even integer is a sum of 2 primes. \\
        \textit{True for all positive numbers up to $4 \cdot 10^{18}$}
    \end{proposition}
    \begin{definition}
        Let $a$ and $b$ be some integers not both zero.
        \begin{enumerate}
            \item The largest $d$ so that $d \mid a$ and $d \mid b$. The largest $d$ so that 
            $d \mid a$ and $d \mid b$ is the greatest common divisor of $a$ and $b$. 
            (We denote it $\gcd(a, b)$.)
            \item The least common multiple of positive integers $a$ and b is the smallest 
            positive integer divisible by both $a$ and $b$ ($\lcm(a, b)$).
        \end{enumerate}
        \begin{enumerate}
            \item \begin{gather*}
                \gcd(a, b) \mid a \\
                \gcd(a, b) \mid b \\
            \end{gather*}
            Moreover, if $d$ is any other common divisor for $a$ and $b$:
            \begin{gather*}
                \begin{aligned}
                    d \mid a \\
                    d \mid b
                \end{aligned}
                \implies d \mid \gcd(a, b)
            \end{gather*}
            \item \begin{gather*}
                a \mid \lcm(a, b) \\
                b \mid \gcd(a, b) \\
            \end{gather*}
            Moreover, if $k$ is any other common multiple for $a$ and $b$:
            \begin{gather*}
                \begin{aligned}
                    a \mid k \\
                    b \mid k
                \end{aligned}
                \implies \lcm(a, b) \mid k
            \end{gather*}
        \end{enumerate}
    \end{definition}
    \begin{theorem}
        $a, b \in \N$
        \[\lcm(a, b) \cdot \gcd(a, b) = a \cdot b\]
    \end{theorem}
    \begin{example}
        What is the $\gcd(24, 36)$?
        \begin{align*}
            \{d \mid d \mid 24\} &= \{1, 2, 3, 4, 6, 8, 12, 24\} \\
            \{d \mid d \mid 36\} &= \{1, 2, 3, 4, 6, 9, 12, 18, 36\} \\
            \gcd(24, 36) &= 12
        \end{align*}
    \end{example}
    \begin{definition}
        \hfill
        \begin{enumerate}
            \item We say that $a$ and $b$ are relatively prime (or coprime) if their greatest 
            common divisor is 1.
            \item The integers $a_1, \dots, a_n$ are pairwise relatively prime if 
            $\gcd(a_i, a_j) = 1$ for any $1 \le i \le n$ and $1 \le j \le n$.
        \end{enumerate}
    \end{definition}
    \begin{example}
        \begin{enumerate}
            \item 10, 17, and 21 are pairwise relatively prime
            \item 10, 19, and 24 are not pairwise relatively prime because $\gcd(10, 24) \ne 1$.
        \end{enumerate}
    \end{example}
    \begin{proof}
        for \textbf{Theorem 0.6} \\
        Nautral numbers $a$ and $b$ enjoy a unique prime factorization.
        \begin{gather*}
            \begin{aligned}
                a = p_1^{a_1} p_2^{a_2} \dots p_n^{a_n} \\
                b = p_1^{b_1} p_2^{b_2} \dots p_n^{b_n}
            \end{aligned} \\
            p_i \text{ are prime numbers} \\
            0 \le a_i, b_i \in \N \\
            \begin{aligned}
                \gcd(a, b) = p_1^{\min(a_1, b_1)} p_2^{\min(a_2, b_2)} \dots p_n^{\min(a_n, b_n)} \\
                \lcm(a, b) = p_1^{\max(a_1, b_1)} p_2^{\max(a_2, b_2)} \dots p_n^{\max(a_n, b_n)}
            \end{aligned}
        \end{gather*}
        For any 2 numbers $a, b$
        \[\min(a, b) + \max(a, b) = a + b\]
        \begin{gather*}
            \gcd(a, b) \cdot \lcm(a, b) \\
            \left(p_1^{\min(a_1, b_1)} p_2^{\min(a_2, b_2)} \dots p_n^{\min(a_n, b_n)}\right)
            \left(p_1^{\max(a_1, b_1)} p_2^{\max(a_2, b_2)} \dots p_n^{\max(a_n, b_n)}\right) \\
            p_1^{\min(a_1, b_1) + \max(a_1, b_1)} p_2^{\min(a_2, b_2) + \max(a_2, b_2)} 
                \dots p_n^{\min(a_n, b_n + \max(a_n, b_n))} \\
            p_1^{a_1 + b_1} p_2^{a_2 + b_2} \dots p_n^{a_n + b_n} = a \cdot b \qed
        \end{gather*}
    \end{proof}
    \begin{lemma}
        Let $a = b \cdot q + r \quad a, b \in \Z$.
        \begin{gather*}
            q, r \in \Z \qquad 0 \le r < b \\
            \gcd(a, b) = \gcd(b, r)
        \end{gather*}
        \begin{proof}
            Let $d = \gcd(a, b)$, then $d \mid a$ and $d \mid b$.
            \begin{enumerate}
                \item \begin{gather*}
                    \begin{cases}
                        a = b \cdot q + r \\
                        d \mid a \\
                        d \mid b
                    \end{cases}
                    \implies d \mid a - b \cdot q = r \\
                    \therefore d \mid r \\
                    \implies \gcd(a, b) \mid b \wedge \gcd(a, b) \mid r \\
                    \implies \gcd(a, b) \mid \gcd(b, r)
                \end{gather*}
                \item Take $k = \gcd(b, r)$ \\
                \[\begin{cases}
                    k \mid b \\
                    k \mid r
                \end{cases}
                \implies k \mid b \cdot q + r = a
                \]
                but $a = b \cdot q + r \implies k \mid a$. \\
                Therefore, $k \mid a$ and $k \mid b$ \\
                so $k \mid \gcd(a, b) \implies \gcd(b, r) \mid \gcd(a, b)$
            \end{enumerate}
            \begin{enumerate}
                \item Part 1 \(\implies \gcd(a, b) \mid \gcd(b, r)\)
                \item Part 2 \(\implies \gcd(b, r) \mid \gcd(a, b)\)
            \end{enumerate}
            \begin{gather*}
                \begin{rcases*}
                    x \mid y \\
                    y \mid x
                \end{rcases*} \quad x = y \\
                x, y \in \Z \\
                x, y \ne 0 \\
                \begin{aligned}
                    y &= ax \\
                    x &= by
                \end{aligned} \implies a, b \in \{\pm 1\} \\
                \begin{aligned}
                    x = abx \\
                    x(1 - ab) = 0
                \end{aligned} \implies a \cdot b = 1
            \end{gather*}
        \end{proof}
    \end{lemma}
    There are three ways to compute $\gcd(a, b)$.
    \begin{enumerate}
        \item List factors
        \item Find prime factorization
        \item Euclidean Algorithm
    \end{enumerate}
    \begin{corollary}
        Euclidean Algorithm \\
        Knowing $\gcd(a, b)$, you know $\lcm(a, b)$.
        Suppose we have $a, b \in \Z$ such that $a \ge b$. 
        Apply division algorithm $\implies q_i \in \Z$.
        \begin{align*}
            r_0 = a &\qquad\qquad r_0 = r_1 \cdot q_2 + r_2 & 0 \le r_2 < r_1 \\
            r_1 = b &\qquad\qquad r_1 = r_2 \cdot q_2 + r_3 & 0 \le r_3 < r_2 \\
            &\qquad\qquad \vdots \\
            &\qquad\qquad r_{n-2} = r_{n-1} \cdot q_{n-1} + r_n \\
            &\qquad\qquad r_{n-1} = r_n \cdot q_n
        \end{align*}
        $a, b \in \N$
        \[\gcd(a, b) = r_n\]
    \end{corollary}
    \begin{proof}
        Lemma 0.7 + Relations 3
        \begin{align*}
            \then \gcd(a, b) &= \gcd(r_0, r_1) \\
            &= \gcd(r_1, r_2) \\
            &= \gcd(r_2, r_3) \\
            &= \gcd(r_n, 0) = r_n
        \end{align*}
    \end{proof}
    \begin{theorem}
        Bézout theorem \\
        If $a, b \in \Z_+$, then there exist integers $s$ and $t$ so that 
        $a \cdot s + b \cdot t = \gcd(a, b)$.
    \end{theorem}
    \begin{corollary}
        If $a, b \in \N$ are coprime, then $\gcd(a, b) = 1$. \\
        $\implies s$ and $t$ so that $\in \Z$
        \[a \cdot s + b \cdot t = 1\]
    \end{corollary}
    \begin{lemma}
        If $a, b, c \in \Z_+$ such that 
        \[\begin{cases}
            a \mid b \cdot c \\
            \gcd(a, b) = 1
        \end{cases}\]
        Then $a \mid c$.
    \end{lemma}
    \begin{proof}
        By Bézout's theorem, $\gcd(a, b) = 1 \implies \exists m, n \in \Z$ such that 
        \begin{gather*}
            \begin{aligned}
                am + bn &= 1 \\
                amc + bnc &= c
            \end{aligned} \\
            \begin{aligned}
                a &\mid amc \\
                a &\mid bc
            \end{aligned} \quad \implies \quad a \mid amc + bnc = c \qed
        \end{gather*}
    \end{proof}
    \begin{corollary}
        $p$ is prime and 
        \[p \mid a_1 a_2 \dots a_n \qquad a_i \in \Z\]
        Then $\exists i = \overline{1, n}$ so that $p \mid a_i$
        \[\begin{aligned}
            p &\mid b \cdot c \\
            p &\nmid b \iff \gcd(p, b) = 1
        \end{aligned} \qquad \text{then} \quad p \mid c\]
    \end{corollary}
\end{document}
