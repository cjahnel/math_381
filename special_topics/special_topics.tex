\documentclass[letterpaper, 12pt]{article}
\usepackage{amsmath, amssymb}
\usepackage{mathtools}

\newtheorem{theorem}{Theorem}[section]
\newtheorem{lemma}[theorem]{Lemma}
\newtheorem{proposition}[theorem]{Proposition}
\newtheorem{corollary}[theorem]{Corollary}

\newenvironment{proof}[1][Proof]{\begin{trivlist}
\item[\hskip \labelsep {\bfseries #1}]}{\end{trivlist}}
\newenvironment{definition}[1][Definition]{\begin{trivlist}
\item[\hskip \labelsep {\bfseries #1}]}{\end{trivlist}}
\newenvironment{example}[1][Example]{\begin{trivlist}
\item[\hskip \labelsep {\bfseries #1}]}{\end{trivlist}}
\newenvironment{remark}[1][Remark]{\begin{trivlist}
\item[\hskip \labelsep {\bfseries #1}]}{\end{trivlist}}

\newcommand{\qed}{\quad \blacksquare}
\newcommand{\keyword}[1]{\textbf{#1}}
\newcommand{\then}{\rightarrow}
\newcommand{\bithen}{\leftrightarrow}
\DeclareMathOperator{\lcm}{lcm}
\DeclareMathOperator{\Dom}{Dom}

\newcommand{\N}{\mathbb{N}}
\newcommand{\Z}{\mathbb{Z}}
\newcommand{\Q}{\mathbb{Q}}
\newcommand{\R}{\mathbb{R}}
\newcommand{\C}{\mathbb{C}}
\newcommand{\0}{\emptyset}
\newcommand{\power}{\mathcal{P}}

\newcommand{\divby}{\; \vdots \;}

\title{MATH 381 Special Topics}
\author{Prof. Olivia Dumitrescu}
\date{22 April 2024}

\begin{document}
    \maketitle
    \[\sum_{n=1}^{\infty} \frac{1}{n} \quad \text{diverges}\]
    \[\sum_{n=1}^{\infty} \frac{1}{n^2} \quad \text{converges}\]
    Let $s \in \R$.
    \[f(s) = \sum_{n=1}^{\infty} \frac{1}{n^s} \quad \begin{cases*}
        \text{converges if } s > 1 \\
        \text{diverges if } s \le 1
    \end{cases*}\]
    Let $s \in \C$. $\zeta : \C \to \C$
    \[\zeta(s) = \sum_{n=1}^{\infty} \frac{1}{n^s}\]
    The zeta function extends the real function $f$ by analytic continuation 
    i.e. $\zeta(s) = f(s)$ wherever $s \in \R \quad s > 1$.
    \begin{theorem}
        There is exactly one way to extend over $\C$ the real function $f(s) \quad s > 1$.
    \end{theorem}
    \subsection*{Basel Problem - Bernoulli}
    Euler proved 
    \[1 + \frac{1}{2^2} + \frac{1}{3^2} + \frac{1}{4^2} + \dots 
    = \frac{\pi^2}{6}\]
    with which we can actually compute 
    \[\sum_{n=1}^{\infty} \frac{1}{n^{2s}} \qquad \begin{aligned}
        s > 1 \\
        s \in \Z
    \end{aligned}\]
    \[\sin : \R \to \R\]
    has a Maclaurin series
    \begin{gather*}
        \sin x = x - \frac{x^3}{3!} + \frac{x^5}{5!} - \frac{x^7}{7!} + \frac{x^9}{9!} + \dots \\
        e^x = \sum_{n=0}^{\infty} \frac{x^n}{n!} = 1 + x + \frac{x^2}{2!} + \frac{x^3}{3!} + \frac{x^4}{4!} + \dots \\
        R = \infty \\
        f(x) \simeq \sum_{n=0}^{\infty} \frac{f^{(n)}(x_0)(x-x_0)^n}{n!} \qquad \begin{aligned}
            |x-x_0|< R \\
            x_0 \in \R
        \end{aligned}
    \end{gather*}
    \begin{align*}
        i^2 &= -1 \\
        e^{i\pi} &= -1 \qquad \text{Euler's formula} \\
        e^{i\theta} &= \cos \theta + i \sin \theta
    \end{align*}
    Take $P(x) : \R \to \R$
    \begin{align*}
        P(a) &= 0 \text{ for } a \in \R \\
        \text{then } P(x) &= (x-a)Q(x)
    \end{align*}
    $\sin x$ has $x = k\pi \qquad k \in \Z$
    \[\sin(k\pi) = 0\]
    \[\sin x = (x - \pi)\frac{\sin x}{x-\pi}\]
    \[\lim_{x \to \pi} \frac{\sin x}{x-\pi} = \frac{\cos(\pi)}{1} = -1\]
    \[h(x) = \begin{cases}
        \dfrac{\sin x}{x-\pi}, & x \ne \pi \\
        -1, & x = \pi
    \end{cases}\]
    \begin{gather*}
        \frac{\sin x}{x(x-\pi)(x+\pi)} \\
        \frac{\sin x}{\prod_{k \in \Z} (x - k\pi)} = 1 \qquad k \in \Z \\
        \frac{1}{-k\pi} (x - k\pi) = 1 - \frac{x}{k\pi} \xrightarrow{k\to\infty} 1-0=1
    \end{gather*}
    \[
        \frac{\sin x}{(1+\frac{x}{2\pi})(1+\frac{x}{\pi})x(1-\frac{x}{\pi})(1-\frac{x}{2\pi})}
        \to 1
    \]
    \begin{align*}
        \sin x &= x \prod_{k \in \Z\setminus\{0\}} \left(1 - \frac{x}{k\pi}\right) \\
        &= x \prod_{k = 1}^{\infty} \left(1 - \frac{x^2}{k^2\pi^2}\right) \\
        &= x - \frac{x^3}{\pi^2} (1 + \frac{1}{4} + \frac{1}{9} + \frac{1}{16}) + 
        + \frac{x^5}{\pi^4}(\dots)
    \end{align*}
    Coeffiencts of $x^3$ in $R_k(x) = x \prod_{n=1}^{k} \left(1 - \dfrac{x^2}{n^2\pi^2}\right)$
    \begin{enumerate}
        \item \[\sum_{n=1}^{k}\frac{1}{n^2\pi^2}
        = -\frac{1}{\pi^2}\left(\sum_{n=1}^{\infty}\frac{1}{n^2}\right)\]
        \item \[\sin x = x - \frac{x^3}{3!} + \frac{x^5}{5!}\]
        \[-\frac{1}{3!} = -\frac{1}{\pi^2} \left(\sum_{n=1}^{\infty} \frac{1}{n^2}\right)
        \implies \sum_{n=1}^{\infty} \frac{1}{n^2} = \frac{\pi^2}{6}\]
    \end{enumerate}
    \subsection*{Bernoulli Numbers}
    \[S_k(n) = 1^k + 2^k + 3^k + \dots + n^k\]
    \begin{align*}
        S_0(n) = n &\qquad k = 0 \\
        S_1(n) = \frac{n(n+1)}{2} &\qquad k = 1 \\
        S_2(n) = \frac{n(n+1)(2n+1)}{6} &\qquad k = 2 \\
        S_3(n) = \left(\frac{n(n+1)}{2}\right)^2 &\qquad k = 3 \\
        S_4(n) = \frac{n(n+1)(2n+1)(3n^2+3n-1)}{20} &\qquad k = 4 \\
        S_5(n) = \frac{n^2(n+1)^2(2n^2+2n-1)}{12} &\qquad k = 5
    \end{align*}
    Faulhaber calculated all numbers $k \le 17$.
    \subsection*{Bernoulli Formula}
    \[\sum_{i=1}^{n} i^k = \frac{1}{k+1} \sum_{j=0}^{k} (-1)^j \binom{k+1}{j} B_j \cdot n^{k+1-j}\]
    $B_j = $ Bernoulli numbers \\
    \begin{align*}
        B_0 &= 1 \\
        B_1 &= \frac{-1}{2} \\
        B_2 &= \frac{1}{6} \\
        B_3 &= 0 \\
        B_4 &= \frac{-1}{30} \\
        B_5 &= 0
    \end{align*}
    $B_\text{odd} = 0$ besides $k = 1$ \\
    Numerators of $B_{2n} = 1, -1, -1, 1, -1, 5$ \\
    Denominators of $B_{2n} = 1, 6, 30, 42, 30, 66$
    \[\sum_{m=1}^{n} (m^k-(m-1)^k) = n^k\]
    \begin{gather*}
        m^k-(m-1)^k = \binom{k}{1}m^{k-1} + \binom{k}{2}m^{k-2} + \dots + (-1)^{k+1}m^0 \\
        \sum_{m=1}^{n} \left[\binom{k}{1}m^{k-1} 0 \binom{k}{2}m^{k-2} + \dots 
        + (-1)^{k+1}m^0\right] = n^k
    \end{gather*}
    \begin{align*}
        k = 0 &\qquad S_0 = n \\
        k = 1 &\qquad S_0 = n \\
        k = 2 &\qquad \sum_{m=1}^{n} \left[\binom{2}{1}m-\binom{2}{2}1\right] = n^2 \\
        &\qquad \sum_{m=1}^{n} (2m-1) = 2S_1 - S_0 = n^2 \\
        k =3 &\qquad 
    \end{align*}
    \subsection*{Taylor Series}
    \begin{align*}
        \sin x &= \sum_{k=0}^{\infty} (-1)^k \frac{x^{2k+1}}{(2k+1)!} \\
        \cos x &= \sum_{k=0}^{\infty} (-1)^k \frac{x^{2k}}{(2k)!} \\
        \tan x &= \sum_{k=1}^{\infty} (-1)^k b_{2k} (1-2^k) \frac{(2x)^{2k-1}}{(2k)!} \\
        x \cot x &= 1 + \sum_{k=1}^{\infty} (-1)^k b_{2k} \frac{(2x)^{2k}}{(2k)!}
        \intertext{Euler}
        \sin x &\cong A \prod_{k\in \Z} (x-k\pi) \\
        &= x \prod_{k \in \Z \setminus \{0\}} \left(1 - \frac{x}{k\pi}\right) \\
        &= x \prod_{k \in \N} \left(1 - \frac{x^2}{k^2\pi^2}\right) \\
        &= x\left(1-\frac{x^2}{\pi^2}\right)\left(1-\frac{x^2}{4\pi^2}\right)
        \left(1-\frac{x^2}{9\pi^2}\right)\left(1-\frac{x^2}{16\pi^2}\right)\dots \\
        &= x \left[1-x^2\left(\frac{1}{\pi^2}+ \frac{1}{4\pi^2} + \frac{1}{9\pi^2} + \frac{1}{16\pi^2} 
        + \dots\right)\right]
    \end{align*}
    Euler
    \begin{align*}
        \zeta(2) &= \frac{\pi^2}{6} \\
        \zeta(4) &= \frac{\pi^4}{90} \\
        \zeta(6) &= \frac{\pi^6}{945}
    \end{align*}
    \[\zeta(s) = \prod_{p} \frac{1}{1 - \frac{1}{p^s}}\]
\end{document}
