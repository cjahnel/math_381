\documentclass[letterpaper, 12pt]{article}
\usepackage{amsmath, amssymb}
\usepackage{mathtools}

\newtheorem{theorem}{Theorem}[section]
\newtheorem{lemma}[theorem]{Lemma}
\newtheorem{proposition}[theorem]{Proposition}
\newtheorem{corollary}[theorem]{Corollary}

\newenvironment{proof}[1][Proof]{\begin{trivlist}
\item[\hskip \labelsep {\bfseries #1}]}{\end{trivlist}}
\newenvironment{definition}[1][Definition]{\begin{trivlist}
\item[\hskip \labelsep {\bfseries #1}]}{\end{trivlist}}
\newenvironment{example}[1][Example]{\begin{trivlist}
\item[\hskip \labelsep {\bfseries #1}]}{\end{trivlist}}
\newenvironment{remark}[1][Remark]{\begin{trivlist}
\item[\hskip \labelsep {\bfseries #1}]}{\end{trivlist}}

\newcommand{\qed}{\quad \blacksquare}
\newcommand{\keyword}[1]{\textbf{#1}}
\newcommand{\then}{\rightarrow}
\newcommand{\bithen}{\leftrightarrow}
\DeclareMathOperator{\lcm}{lcm}
\DeclareMathOperator{\Dom}{Dom}

\newcommand{\N}{\mathbb{N}}
\newcommand{\Z}{\mathbb{Z}}
\newcommand{\Q}{\mathbb{Q}}
\newcommand{\R}{\mathbb{R}}
\newcommand{\C}{\mathbb{C}}
\newcommand{\0}{\emptyset}
\newcommand{\power}{\mathcal{P}}

\newcommand{\divby}{\; \vdots \;}

\title{MATH 381 Section 6.4}
\author{Prof. Olivia Dumitrescu}
\date{1 April 2024}

\begin{document}
    \maketitle
    \section*{Section 6.4 The Binomial Theorem}
    \begin{theorem}
        Binonmial Theorem \\
        Let $x, y$ be variables and $n \ge 0$.
        \[\begin{split} (x + y)^n &= \sum_{k=0}^{n} \binom{n}{k} x^k \cdot y^{n-k} \\
            &= \sum_{k=0}^{n} \binom{n}{k} x^{n-k} \cdot y^k \\
            &= \binom{n}{0}x^n + \binom{n}{1}x^{n-1}y + \dots 
            + \binom{n}{n-1}xy^{n-1} + \binom{n}{n}y^n \\
            &= x^n + nx^{n-1}y + \frac{n(n-1)}{2}x^{n-2}y^2 + \dots 
            + nxy^{n-1} + y^n
        \end{split}\]
    \end{theorem}
    \begin{corollary}
        \[2^n = \sum_{k=0}^{n} \binom{n}{k} = 
        \binom{n}{0} + \binom{n}{1} + \binom{n}{2} + \dots + \binom{n}{n-1} + \binom{n}{n}\]
        $x = y = 1$ \\
        $|\power(S)|$
    \end{corollary}
    $n \in \N$
    \[(1 + x)^n = \sum_{k=0}^{n} \binom{n}{k} x^k\]
    Taylor Series $\qquad x_0 \in \Dom f$
    \[f(x) = \sum_{k=0}^{\infty} \frac{f^{(k)}(x_0)}{k!} (x - x_0)^k\]
    Maclaurin Series Expansion if $x_0 = 0$. \\
    $f(x) = (1 + x)^n$ \\
    $m \notin \N, m \in \R$
    \[(1 + x)^m = \sum_{k=0}^{\infty} \binom{m}{k} x^k\]
    \[\binom{m}{k} = \begin{cases}
        \dfrac{m(m-1)(m-2)\dots(m-k+1)}{k!} & k > 0 \\
        1 & k = 0
    \end{cases}\]
    \begin{align*}
        \binom{-2}{3} &= \frac{-2 \cdot -3 \cdot -4}{1 \cdot 2 \cdot 3} = \frac{-4}{1} = -4 \\
        \binom{\frac{1}{2}}{3} &= \frac{\frac{1}{2}(\frac{1}{2}-1)(\frac{1}{2}-2)}{1 \cdot 2 \cdot 3} 
        = \frac{\frac{1}{2} \cdot \frac{-1}{2} \cdot \frac{-3}{2}}{2 \cdot 3} 
        = \frac{\frac{1}{2} \cdot \frac{-1}{2} \cdot \frac{-1}{2}}{2}  = \frac{1}{16}
    \end{align*}
    \begin{align*}
        \binom{-1}{k} &= \frac{-1 \cdot -2 \cdot \dots \cdot (-1 - k + 1)}
        {1 \cdot 2 \cdot \dots \cdot k} = \frac{-1 \cdot -2 \cdot \dots \cdot -k}
        {1 \cdot 2 \cdot 3 \cdot \dots \cdot k} \\
        &= (-1)^k = \begin{cases}
            -1, & k \text{ is odd} \\
            +1, & k \text{ is even}
        \end{cases}
    \end{align*}
    Use binomial formula
    \[\frac{1}{1-x} = (1-x)^{-1} = \sum_{k=0}^{\infty} \binom{-1}{k} (-x)^k\]
    \[\frac{1}{1-x} = 1 + x + x^2 + x^3 + \dots + x^n + \dots \qquad |x| < 1\]
    \begin{align*}
        \frac{1}{1-x} &= (1-x)^{-1} \\
        &= \sum_{k=0}^{\infty} \binom{-1}{k} (-x)^k \\
        &= \sum_{k=0}^{\infty} (-1)^k (-1)^k \cdot x^k = \sum_{k=0}^{\infty} x^k
    \end{align*}
    \begin{corollary}
        $n \in \N$
        \[\sum_{k=0}^{n} (-1)^k \binom{n}{k} = 0\]
        \[\binom{n}{0} - \binom{n}{1} + \binom{n}{2} + \dots + (-1)^n \binom{n}{n} = 0\]
    \end{corollary}
    \begin{proof}
        $x = 1, y = -1$
        \[\sum_{k=0}^{n} (-1)^k \binom{n}{k} = (1-1)^n = 0\]
    \end{proof}
    \begin{corollary}
        \[\sum_{k=0}^{n} \binom{n}{k} 2^k = 3^n\]
    \end{corollary}
    \begin{proof}
        $x = 2, y = 1$
        \[\sum_{k=0}^{n} \binom{n}{k} 2^k = (2+1)^n = 3^n\]
    \end{proof}
    \begin{corollary}
        \[n \cdot 2^{n-1} = \binom{n}{1} + 2\binom{n}{2} + \dots + n\binom{n}{n} 
        = \sum_{k=1}^{n} k \binom{n}{k}\]
    \end{corollary}
    \begin{proof}
        \[(1+x)^n = \sum_{k=0}^{n} \binom{n}{k} x^k\]
        Differentiate (with respect to $x$) [possible $\because$ we expanded the domain to $\R$]
        \[n(1+x)^{n-1} = \sum_{k=1}^{n} k \binom{n}{k} x^{k-1}\]
        Let $x=1$.
        \[n \cdot 2^{n-1} = \sum_{k=1}^{n} k \binom{n}{k}\]
    \end{proof}
    \begin{remark}
        If $n \in \N, k \in \N, k \ge n + 1$
        \[\binom{n}{k} = 0\]
        \[\binom{n}{n+1} = \frac{n(n-1)\dots(n-(n+1)+1)}{(n+1)!} 
        = \frac{n(n-1)\dots(0)}{(n+1)!} = 0\]
    \end{remark}
    \begin{theorem}
        Pascal's Identity \\
        Let $n, k \in \N \qquad k \le n$
        \[\binom{n+1}{k} = \binom{n}{k-1} + \binom{n}{k}\]
    \end{theorem}
    \begin{proof}
        \begin{align*}
            \binom{n+1}{k} &= \frac{(n+1)n(n-1)\dots((n+1)-k+1)}{k!} \\
            \binom{n}{k-1} &= \frac{n(n-1)\dots(n-(k-1)+1)}{(k-1)!} \\
            \binom{n}{k} &= \frac{n(n-1)\dots(n-k+1)}{k!}
        \end{align*}
        \begin{align*}
            \binom{n+1}{k} &= \binom{n}{k-1} + \binom{n}{k} \\
            \frac{(n+1)!}{k!(n+1-k)!} &= \frac{n!}{(k-1)!(n-k+1)!} + \frac{n!}{k!(n-k)!} \\
            &= \frac{n!}{(k-1)!(n-k)!} \left[\frac{1}{n-k+1}+\frac{1}{k}\right] \\
            &= \frac{n!}{(k-1)!(n-k)!} \left[\frac{k}{k(n-k+1)}+\frac{(n-k+1)}{k(n-k+1)}\right] \\
            &= \frac{n!}{(k-1)!(n-k)!} \left[\frac{k+(n-k+1)}{k(n-k+1)}\right] \\
            &= \frac{n!}{(k-1)!(n-k)!} \cdot \frac{n+1}{k(n-k+1)} \\
            &= \frac{(n+1)!}{k!(n-k+1)!}
        \end{align*}
    \end{proof}
    \begin{theorem}
        Vandermonde's Identity \\
        Let $m, n, r \in \N$. $r \le m, n$
        \[\binom{m+n}{r} = \sum_{k=0}^{r} \binom{m}{r-k} \binom{n}{k}\]
    \end{theorem}
    \begin{proof}
        \begin{align*}
            (1+x)^{m+n} &= (1+x)^m \cdot (1+x)^n \\
            \sum_{r=0}^{m+n} \binom{m+n}{r} x^r 
            &= \left(\sum_{i=0}^{m} \binom{m}{i} x^i\right)
            \left(\sum_{j=0}^{n} \binom{n}{j} x^j\right) \\
            &= \sum_{k=0}^{m+n} \left(\sum_{i+j=k} \binom{m}{i}\binom{n}{j}\right) x^k
            \qquad k = r
        \end{align*}
    \end{proof}
    \begin{corollary}
        $n \in \N$
        \[\binom{2n}{n} = \sum_{k=0}^{n} \binom{n}{k}^2\]
    \end{corollary}
    \begin{proof}
        $m=n=r$
        \[\begin{split}
            \binom{2n}{n} &= \sum_{k=0}^{n} \binom{n}{n-k} \binom{n}{k} \\
            &= \sum_{k=0}^{n} \binom{n}{k}^2
        \end{split}\]
    \end{proof}
    \begin{theorem}
        $n, r \in \N \qquad r \le n$
        \[\binom{n+1}{r+1} = \binom{r}{r} + \binom{r+1}{r} + \binom{r+2}{r} + \dots 
        + \binom{n}{r} = \sum_{j=r}^{n} \binom{j}{r}\]
    \end{theorem}
    \begin{proof}
        \[\begin{alignedat}{3}
            \binom{n+1}{r+1} = \binom{n}{r} + &\binom{n}{r+1} & & \\
            &\binom{n-1}{r} + &\binom{n-1}{r+1} & \\
            & &\binom{n-2}{r} + &\binom{n-2}{r+1} \\
            & & &\binom{r}{r}
        \end{alignedat}\]
    \end{proof}
    \begin{example}
        What is the coefficient of $x^{12} \cdot y^{13}$ in the expansion $(2x-3y)^{35}$?
    \end{example}
    \begin{proof}
        \[(2x-3y)^{25} = \sum_{k=0}^{25} \binom{25}{k}(2x)^{25-k} \cdot (-3y)^k\]
        So, $k = 13$.
        \[\binom{25}{13} \cdot 2^{12} \cdot (-3)^{13}\]
    \end{proof}
\end{document}
