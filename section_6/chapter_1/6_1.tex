\documentclass[letterpaper, 12pt]{article}
\usepackage{amsmath, amssymb}
\usepackage{mathtools}

\newtheorem{theorem}{Theorem}[section]
\newtheorem{lemma}[theorem]{Lemma}
\newtheorem{proposition}[theorem]{Proposition}
\newtheorem{corollary}[theorem]{Corollary}

\newenvironment{proof}[1][Proof]{\begin{trivlist}
\item[\hskip \labelsep {\bfseries #1}]}{\end{trivlist}}
\newenvironment{definition}[1][Definition]{\begin{trivlist}
\item[\hskip \labelsep {\bfseries #1}]}{\end{trivlist}}
\newenvironment{example}[1][Example]{\begin{trivlist}
\item[\hskip \labelsep {\bfseries #1}]}{\end{trivlist}}
\newenvironment{remark}[1][Remark]{\begin{trivlist}
\item[\hskip \labelsep {\bfseries #1}]}{\end{trivlist}}

\newcommand{\qed}{\quad \blacksquare}
\newcommand{\keyword}[1]{\textbf{#1}}
\newcommand{\then}{\rightarrow}
\newcommand{\bithen}{\leftrightarrow}
\DeclareMathOperator{\lcm}{lcm}
\DeclareMathOperator{\Dom}{Dom}

\newcommand{\N}{\mathbb{N}}
\newcommand{\Z}{\mathbb{Z}}
\newcommand{\Q}{\mathbb{Q}}
\newcommand{\R}{\mathbb{R}}
\newcommand{\C}{\mathbb{C}}
\newcommand{\0}{\emptyset}
\newcommand{\power}{\mathcal{P}}

\newcommand{\divby}{\; \vdots \;}

\title{MATH 381 Section 6.1}
\author{Prof. Olivia Dumitrescu}
\date{15 April 2024}

\begin{document}
    \maketitle
    \section*{Section 6.1 The Basics of Counting}
    \subsection*{Counting Functions}
    How many functions from a set of $m$ elements to a set of $n$ elements?
    \[n^m\]
    \subsection*{One-to-One Functions}
    How many 1-to-1 functions from a set with $m$ elements to a set with $n$ elements?
    \[n(n-1)\dots(n-m+1)\]
    \begin{example}
        Use the product rule to show that the no. of different subsets of a finite set $S$ is 
        $2^{|S|}$. $S = \{x_1,\dots,x_n\}$.
        \begin{flushleft}
            Let $S$ = finite set. List the elements of $S$ in arbitrary order. Recall that there 
            is a 1-to-1 correspondence between subsets of $S$ and bitstrings of length $2^{|S|}$.
            \[\text{e.g. } \{x_2, x_3, x_5\} \to \{0, 1, 1, 0, 1, 0,\dots, 0\}\]
            By the product rule, there are $2^{|S|}$ bitstrings of length $|S|$, i.e. 
            $|\power(S)| = 2^{|S|}$
        \end{flushleft}
    \end{example}
    \begin{example}
        How many bitstrings of length 8 start with a 1 bit or end with the two bits 00?
        \[|A_1 \cup A_2| = |A_1| + |A_2| - |A_1 \cap A_2| = 2^7 + 2^6 - 2^5 
        = 2^5 (2^2 + 2 - 1) = 32 \dot 5 = 160\]
    \end{example}
    There are $n \mid d$ ways to do a task if it can be done using a procedure that can be carried 
    out in $n$ ways and for each way $w$, exactly $d$ of the $n$ ways correspond to way $w$.
    \begin{example}
        How many ways are there to seat 4 people at a circular table? (2 sittings are the same if 
        each person has the same left and right neighbors i.e. equal under rotation.)
        \begin{align*}
            4! &= 24 \text{ (to order)} \\
            \frac{4!}{4} &= \frac{24}{4} = 6 \text{ (4 ways to choose a person in seat 1)}
        \end{align*}
    \end{example}
    \begin{example}
        How many different bitstrings of length 4 do not have consecutive 1s?
        \begin{align*}
            2^3 = 8 &= \text{\# of bitstrings that have 2 consecutive 1s} \\
            2^4 = 16 &= \text{\# of total bitstrings} \\
            2^4 - 2^3 = 8 &= \text{\# of bitstrings with no consecutive 1s}
        \end{align*}
    \end{example}
\end{document}
