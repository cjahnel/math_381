\documentclass[letterpaper, 12pt]{article}
\usepackage{amsmath, amssymb}
\usepackage{mathtools}

\newtheorem{theorem}{Theorem}[section]
\newtheorem{lemma}[theorem]{Lemma}
\newtheorem{proposition}[theorem]{Proposition}
\newtheorem{corollary}[theorem]{Corollary}

\newenvironment{proof}[1][Proof]{\begin{trivlist}
\item[\hskip \labelsep {\bfseries #1}]}{\end{trivlist}}
\newenvironment{definition}[1][Definition]{\begin{trivlist}
\item[\hskip \labelsep {\bfseries #1}]}{\end{trivlist}}
\newenvironment{example}[1][Example]{\begin{trivlist}
\item[\hskip \labelsep {\bfseries #1}]}{\end{trivlist}}
\newenvironment{remark}[1][Remark]{\begin{trivlist}
\item[\hskip \labelsep {\bfseries #1}]}{\end{trivlist}}

\newcommand{\qed}{\quad \blacksquare}
\newcommand{\keyword}[1]{\textbf{#1}}
\newcommand{\then}{\rightarrow}
\newcommand{\bithen}{\leftrightarrow}
\DeclareMathOperator{\lcm}{lcm}
\DeclareMathOperator{\Dom}{Dom}

\newcommand{\N}{\mathbb{N}}
\newcommand{\Z}{\mathbb{Z}}
\newcommand{\Q}{\mathbb{Q}}
\newcommand{\R}{\mathbb{R}}
\newcommand{\C}{\mathbb{C}}
\newcommand{\0}{\emptyset}
\newcommand{\power}{\mathcal{P}}

\newcommand{\divby}{\; \vdots \;}

\title{MATH 381 Section 6.3}
\author{Prof. Olivia Dumitrescu}
\date{1 April 2024}

\begin{document}
    \maketitle
    \section*{Section 6.3 Permutations}
    $n \in \N$ \\
    $n! = n(n-1)(n-2)\dots2 \cdot 1$
    \begin{example}
        $S = \{1, 2, 3\}$
        The arragement $\{3, 1, 2\}$ is a premutation of $S$
        \begin{itemize}
            \item 2-permutation of $S$
            \[\{1, 2\} \quad \{2, 1\} 
            \quad \{1, 3\} \quad \{3, 1\} 
            \quad \{2, 3\} \quad \{3, 2\}\]
        \end{itemize}
    \end{example}
    $n = 3, r = 2$ \\
    $P(n, r)$ is the number of $r$-permutations of a set of order $n$.
    \begin{theorem}
        If $n$ is a positive integer and 
        \[1 \le r \le n \qquad r \in \N\]
        \[\begin{cases}
            P(n, r) =  n(n-1)(n-2)\dots(n-r +1) \\
            P(n, 0) = 1 \qquad n \in \N
        \end{cases}\]
    \end{theorem}
    \begin{corollary}
        $n \in \N \qquad 0 \le r \le n \qquad r \in \N$
        \[P(n, r) = \frac{n!}{(n-r)!} = \frac{n(n-1)\dots(n-r +1)(n-r)\dots1}{(n-r)(n-r-1)\dots1}\]
    \end{corollary}
    \begin{example}
        How many permutations of letters $ABCDEFGH$ contain the string $ABC$? \\
        $ABC$ = $X$ \\
        \# of permutations = $XDEFGH$ \\
        i.e. \# rearrangements = $6! = 6 \cdot 5 \cdot 4 \cdot 3 \cdot 2 \cdot 1$
    \end{example}
    \begin{remark}
        $n! = \#$ of arrangements of a set of order $n$.
    \end{remark}
    \begin{definition}
        An $r$-combination of a set of ordern $n$ is an unordered selection of $r$ elements of 
        the set. (Equivalently, it is the number of subsets of order $r$ of a set of order $n$.)
    \end{definition}
    \begin{example}
        \hfill
        \begin{enumerate}
            \item $P(3, 2) = 6$
            \item $C(3, 2) = 3$
        \end{enumerate}
    \end{example}
    \begin{theorem}
        $n \in \N, 0 \le r \le n$
        \begin{enumerate}
            \item \[P(n, r) = \frac{n!}{(n-r)!} = n \cdot (n-1)\dots(n-r+1)\]
            \item \[C(n, r) = \frac{n!}{r!(n-r)!} = \binom{n}{r}\]
        \end{enumerate}
    \end{theorem}
    \begin{corollary}
        $n, r \in \N \qquad 0 \le r \le n$ \\
        Then $C(n, r) = C(n, n-r)$ or \\
        \[\binom{n}{r} = \binom{n}{n-r}\]
    \end{corollary}
\end{document}
