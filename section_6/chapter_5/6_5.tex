\documentclass[letterpaper, 12pt]{article}
\usepackage{amsmath, amssymb}
\usepackage{mathtools}

\newtheorem{theorem}{Theorem}[section]
\newtheorem{lemma}[theorem]{Lemma}
\newtheorem{proposition}[theorem]{Proposition}
\newtheorem{corollary}[theorem]{Corollary}

\newenvironment{proof}[1][Proof]{\begin{trivlist}
\item[\hskip \labelsep {\bfseries #1}]}{\end{trivlist}}
\newenvironment{definition}[1][Definition]{\begin{trivlist}
\item[\hskip \labelsep {\bfseries #1}]}{\end{trivlist}}
\newenvironment{example}[1][Example]{\begin{trivlist}
\item[\hskip \labelsep {\bfseries #1}]}{\end{trivlist}}
\newenvironment{remark}[1][Remark]{\begin{trivlist}
\item[\hskip \labelsep {\bfseries #1}]}{\end{trivlist}}

\newcommand{\qed}{\quad \blacksquare}
\newcommand{\keyword}[1]{\textbf{#1}}
\newcommand{\then}{\rightarrow}
\newcommand{\bithen}{\leftrightarrow}
\DeclareMathOperator{\lcm}{lcm}
\DeclareMathOperator{\Dom}{Dom}

\newcommand{\N}{\mathbb{N}}
\newcommand{\Z}{\mathbb{Z}}
\newcommand{\Q}{\mathbb{Q}}
\newcommand{\R}{\mathbb{R}}
\newcommand{\C}{\mathbb{C}}
\newcommand{\0}{\emptyset}
\newcommand{\power}{\mathcal{P}}

\newcommand{\divby}{\; \vdots \;}

\title{MATH 381 Section 6.5 Generalized Permutations and Combinations}
\author{Prof. Olivia Dumitrescu}
\date{1 April 2024}

\begin{document}
    \maketitle
    \section*{Section 6.5 Permutations with Repetitions}
    \begin{theorem}
        The number of $r$-permutations of a set of $n$-objects with reptition allowed is $n^r$.
    \end{theorem}
    \subsection*{Combinations with Repetitions}
    \begin{theorem}
        There are $\dbinom{n+r-1}{r} = \dbinom{n+r-1}{n-1}$ number of $r$-combinations from a 
        set with $n$ elements when repetition of elements is allowed.
    \end{theorem}
    \begin{remark}
        Recall that Gauss' formula for the sum of a series up to $n$ gives the number of 
        lattice points in triangle.
        \[1+2+3+\dots+n = \frac{n(n+1)}{2} = \binom{n+1}{2}\]
    \end{remark}
    \begin{theorem}
        Say we have a hyperplane in $\R^n$
        \[x_1 + x_2 + x_3 + \dots + x_n = d\]
        Then the number of integer lattice points is 
        \[\binom{d+n}{n} \qquad \dim_k k[x_1,\dots,x_n]\]
        \[\binom{d+2}{2} = \frac{(d+2)(d+1)}{2} = 1 + 2 + \dots + (d+1)\]
        the space of polynomials in $n$ variables $x_1,\dots,x_n$ of degree $\le d$ is a vector 
        space of dimension
        \[\binom{n+d}{n} = \binom{n+d}{d}\]
    \end{theorem}
    \begin{corollary}
        The number of lattice points with non-negative integer coefficients inside the hyperplane is 
        \[\binom{n+d-1}{n-1} = \binom{n+d}{n} - \binom{n+d-1}{n}\]
    \end{corollary}
    \begin{corollary}
        i.e. homogeneous polynomial in $n+1$ variables $x_0,x_1,\dots,x_n$ of total degree $d$ 
        forms a vector space $\binom{n+d}{n}$.
        \[a + bx + cy + dxy + ex^2 + fy^2 \qquad \dim_k k[x, y]_{\le d=2} = \binom{2+2}{2} = 6\]
    \end{corollary}
    \begin{example}
        How many solutions with non-negative integers are there to $x_1+x_2+x_3=11$?
        \begin{enumerate}
            \item \[\binom{11+3}{3} - \binom{10+3}{3} = \binom{13}{2}\]
            \item Number of ways to select 11 items from a set with 3 elements so that \\
            $x_1$ of first element \\
            $x_2$ second element \\
            $x_3$ third element
        \end{enumerate}
    \end{example}
    \subsection*{Permutations with Indistinguishable Objects}
    \begin{example}
        How many different words do we have by rearranging the word SUCCESS?
        \[\binom{7}{3}\binom{4}{2}\binom{2}{1}\binom{1}{1} = \frac{7!}{3!2!}\]
    \end{example}
    \begin{theorem}
        The number of different permutations of $n$ objects \\
        $n_1=$ Indistinguishable objects of type 1 \\
        $n_2=$ Indistinguishable objects of type 2 \\
        \vdots \\
        $n_k=$ Indistinguishable objects of type k
        \[\frac{n!}{n_1!n_2!n_3!\dots n_k!}\]
    \end{theorem}
    \begin{example}
        What is the number of ways to distribute hands of 5 cards to 4 players from a standard 
        deck of 52 cards?
        \[\binom{52}{5}\binom{47}{5}\binom{42}{5}\binom{37}{5}\]
    \end{example}
    \begin{theorem}
        The number of ways to distribute $n$ distinguishable objects into $k$ distinguishable 
        boxes so that $n_i$ objects are in box $i$
        \[\frac{n!}{n_1!n_2!\dots n_k!}\]
    \end{theorem}
    \begin{definition}
        The Striling Numbers \\
        $S(n, j) = $ number of ways to distribute $n$ distinguishable objects into $j$ 
        indistinguishable boxes. A closed formula is not known.
    \end{definition}
    \begin{theorem}
        \[S(n, j) = \frac{1}{j!} \cdot \sum_{i=0}^{j-1} (-1)^i \binom{j}{i} (j-i)^n\]
        so the number of ways to distribute $n$ distinguishable objects into $k$ Indistinguishable 
        boxes equals
        \[\sum_{j=1}^{k} S(n, j) = \sum_{j=1}^{k} \frac{1}{j!} \cdot \sum_{i=0}^{j-1} (-1)^i \binom{j}{i} (j-i)^n\]
    \end{theorem}
\end{document}
