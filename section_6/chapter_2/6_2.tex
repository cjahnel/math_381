\documentclass[letterpaper, 12pt]{article}
\usepackage{amsmath, amssymb}
\usepackage{mathtools}

\newtheorem{theorem}{Theorem}[section]
\newtheorem{lemma}[theorem]{Lemma}
\newtheorem{proposition}[theorem]{Proposition}
\newtheorem{corollary}[theorem]{Corollary}

\newenvironment{proof}[1][Proof]{\begin{trivlist}
\item[\hskip \labelsep {\bfseries #1}]}{\end{trivlist}}
\newenvironment{definition}[1][Definition]{\begin{trivlist}
\item[\hskip \labelsep {\bfseries #1}]}{\end{trivlist}}
\newenvironment{example}[1][Example]{\begin{trivlist}
\item[\hskip \labelsep {\bfseries #1}]}{\end{trivlist}}
\newenvironment{remark}[1][Remark]{\begin{trivlist}
\item[\hskip \labelsep {\bfseries #1}]}{\end{trivlist}}

\newcommand{\qed}{\quad \blacksquare}
\newcommand{\keyword}[1]{\textbf{#1}}
\newcommand{\then}{\rightarrow}
\newcommand{\bithen}{\leftrightarrow}
\DeclareMathOperator{\lcm}{lcm}
\DeclareMathOperator{\Dom}{Dom}

\newcommand{\N}{\mathbb{N}}
\newcommand{\Z}{\mathbb{Z}}
\newcommand{\Q}{\mathbb{Q}}
\newcommand{\R}{\mathbb{R}}
\newcommand{\C}{\mathbb{C}}
\newcommand{\0}{\emptyset}
\newcommand{\power}{\mathcal{P}}

\newcommand{\divby}{\; \vdots \;}

\title{MATH 381 Section 6.2}
\author{Prof. Olivia Dumitrescu}
\date{15 April 2024}

\begin{document}
    \maketitle
    \section*{Section 6.2 The Pigeonhole Principle}
    If $k \in \Z_+$ and $k + 1$ or more objects are placed in $k$ boxes, then $\exists!$ at least 
    one box containing at least 2 objects.
    \begin{corollary}
        A function from a set of $k+1$ or more elements to a set with $k$ elements is not 1-to-1.
    \end{corollary}
    \subsection*{The Generalized Pigeonhole Principle}
    If $N$ objects are placed in $k$ boxes, then $\exists$ at elast one box containing at least 
    \[\lceil \tfrac{N}{k} \rceil \text{ objects.}\]
    \begin{example}
        During a month of 30 days, a team plays at least 1 game per day but not more than 45 games. 
        Show that there exists a period of some number of consecutive days where the team must play 
        exactly 14 games. \\
        \\
        At most 15 extra games to play, so there are at most 15 days where the team plays at least 
        2 games per day. \\
        Let $a_j$ be the number of games played before day $j$ of the month.
        \begin{gather*}
            \intertext{Since at least one game per day}
            a_1 < a_2 < \dots < a_{30} \\
            15 \le a_j + 14 \le 45 + 14 = 59 \\
            a_1 + 14 < a_2 + 14 < \dots < a_{30} + 14 \\
            \intertext{Consider 60 numbers}
            \begin{cases}
                a_1 < a_2 < \dots < a_{30} \\
                a_1 + 14 < a_2 + 14 < \dots < a_{30} + 14
            \end{cases}
            \intertext{So 60 positive integers all $\le$ 59 $\implies \exists$ two of them 
            that are equal.}
            \intertext{However, $a_i \ne a_j$ since $a_i < a_j$ and also $a_i + 14 \ne a_j + 14 
            \quad \forall i, j$}
            \implies \exists i, j \text{ so that } a_i = a_j + 14.
            \intertext{i.e. in between day $j$ and day $i$ the team plays exactly 14 games}
        \end{gather*}
    \end{example}
    \begin{example}
        Show that among $n+1$ positive integers not exceeding $2n$, there must be an integer that 
        divides one of the other integers. \\
        \\
        Write each of the $n+1$ integers $a_1,a_2,\dots,a_n$ as a power of 2 times an odd integer
        \[a_j = 2^{k_j} \cdot q_j \quad \text{for } j = 1,2,\dots,n+1\]
        \[\begin{cases}
            k_j = \text{ non-negative integer} \\
            q_j = \text{ odd}
        \end{cases}\]
        The integers $q_1,\dots,q_{n+1}$ are all odd positive integers less than $2n$. \\
        There are only $n$ positive integers less than $2n \implies$ i.e. at least 2 integers 
        $q_1,\dots,q_{n+1}$ must be equal. \\
        Therefore, there exists distinct integers $i, j$ so that $q_i = q_j$.
        \[q = \text{ common value of } q_i, q_j\]
        \[\begin{cases}
            a_i = 2^{k_i} \cdot q & \text{if } k_i < k_j \implies a_i \mid a_j \\
            a_j = 2^{k_j} \cdot q & \text{if } k_j < k_i \implies a_j \mid a_i
        \end{cases}\]
    \end{example}
    \begin{theorem}
        Every sequence of $n^2 + 1$ distinct numbers contains a subsequence of length $n+1$ 
        that is either strictly increasing or strictly decreasing.
    \end{theorem}
    \begin{proof}
        Let $a_1, a_2, \dots, a_{n^2 + 1}$ be a sequence of $n^2 + 1$ distinct real numbers. 
        For each term $a_k \rightsquigarrow$ associate an ordered pair $(i_k, d_k)$ where 
        \begin{itemize}
            \item $i_k =$ length of the longest increasing subsequence starting at $a_k$
            \item $d_k =$ length of the longest decreasing subsequence starting at $a_k$
        \end{itemize}
        Assume by contradiction that there are no increasing or decreasing sequences of length 
        $n+1$. Then $i_k, d_k \in \N; \quad i_k, d_k \le n$. \\
        For any $k=1,2,\dots,n^2+1$ then both $i_k \le n$ and $d_k \le n$. Then by the product 
        rule there are $n^2$ possible ordered pairs for $(i_k, d_k)$ of a total of 
        $k=1,2,\dots,n^2+1$. Then two of them must be equal i.e. $\exists s < t$ so that 
        $i_s = i_t \wedge d_s = d_t$. \\
        We prove this is impossible. Since the terms of the sequence are all distinct, either 
        $a_s < a_t$ or $a_s > a_t$. \\
        \\
        WLOG assume $\begin{cases}
            a_s < a_t \\
            i_s = i_t
        \end{cases}$ \\
        $\implies$ an increasing subsequence of length $i_t+1$ can be built starting from $a_s$, 
        followex by an increasing subsequence of length $i_t$ beginning at $a_t$. Contradiction.
    \end{proof}
    \begin{example}
        Assume that in a group of 6 people each pair of individuals consists of two friends or 
        two enemies. Show that there are either 3 mutual friends or mutual enemies in a group. \\
        \\
        $(A, B)$ are either friends or enemies. \\
        Let $A =$ one person in the group; and 5 other people remaining.
        \[5 = 2+3\]
        So out of these 5 people either 3 or more are friends of $A$ or 3 or more are enemies of $A$. \\
        Because 5 objects are divided into 2 sets, one of the sets must have at least 
        $\lceil \frac{5}{2} \rceil = 3$ elements. \\
        Suppose $\{B,C,D\}$ are friends of $A$. \\
        If any one of these 3 individuals are friends, then these three and $A$ form a group of 
        3 mutual friends. Otherwise
        \[\{B,C,D\} = \text{ set of 3 mutual enemies}\]
    \end{example}
    \subsection*{The Ramsey Number}
    $R(m, n) = $ minimum  people at a party $\quad m, n \in \N; \quad m, n \ge 2$ \\
    so that there are either $m$ mutual friends or $n$ mutual enemies assuming every 2 people 
    are either friends or enemies.
    \[R(3, 3) = 6\]
    Properties:
    \begin{align*}
        R(m, n) &= R(n, m) \\
        R(2, n) &= n \qquad \forall n \ge 2
    \end{align*}
    Known for $R(m, n) \quad 3 \le m \le n$:
    \begin{gather*}
        R(4, 4) = 16
    \end{gather*}
    For any other numbers, only bounds are known:
    \[42 \le R(5, 5) \le 49\]
\end{document}
