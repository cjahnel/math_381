\documentclass[letterpaper, 12pt]{article}
\usepackage{amsmath, amssymb}

\newtheorem{theorem}{Theorem}[section]
\newtheorem{lemma}[theorem]{Lemma}
\newtheorem{proposition}[theorem]{Proposition}
\newtheorem{corollary}[theorem]{Corollary}

\newenvironment{proof}[1][Proof]{\begin{trivlist}
\item[\hskip \labelsep {\bfseries #1}]}{\end{trivlist}}
\newenvironment{definition}[1][Definition]{\begin{trivlist}
\item[\hskip \labelsep {\bfseries #1}]}{\end{trivlist}}
\newenvironment{example}[1][Example]{\begin{trivlist}
\item[\hskip \labelsep {\bfseries #1}]}{\end{trivlist}}
\newenvironment{remark}[1][Remark]{\begin{trivlist}
\item[\hskip \labelsep {\bfseries #1}]}{\end{trivlist}}

\newcommand{\qed}{\quad \blacksquare}
\newcommand{\keyword}[1]{\textbf{#1}}
\newcommand{\then}{\rightarrow}
\newcommand{\bithen}{\leftrightarrow}

\newcommand{\N}{\mathbb{N}}
\newcommand{\Z}{\mathbb{Z}}
\newcommand{\Q}{\mathbb{Q}}
\newcommand{\R}{\mathbb{R}}
\newcommand{\C}{\mathbb{C}}
\newcommand{\0}{\emptyset}
\newcommand{\power}{\mathcal{P}}

\title{MATH 381 Section 5.2}
\author{Prof. Olivia Dumitrescu}
\date{1 March 2024}

\begin{document}
    \maketitle
    \section*{Strong Induction and Well-Ordering}
    Strong Induction
    \begin{enumerate}
        \item $P(1)$ holds
        \item $\forall 1 \le j \le k \quad P(j) \then P(k + 1)$
    \end{enumerate}
    Then we conclude $P(n)$ holds for $n \ge 1$.
    \begin{definition}
        A number is called \keyword{prime} if it is non-unital and the only divisors of the 
        number are 1 and itself.
        \[1 \mid p \qquad p \mid p\]
        ($p = 1$ is not prime by definition)
    \end{definition}
    \begin{example}
        Show that if $n \in \Z$, then either $n$ is prime or $n$ can be written as a product 
        of primes. (In fact, this decomposition is unique.)
    \end{example}
    \begin{proof} Strong Induction
        \begin{align*}
            P(n) &\quad n \text{ can be written as a product of primes.} \\
            n \ge 2 &
        \end{align*}
        \begin{enumerate}
            \item $P(2), P(3), P(4), P(5), P(6), P(8)$
            \item Assume $P(j)$ holds for any $1 \le j \le n$. \\
            We claim $P(n+1)$ holds.
            \begin{flushleft}
                Let $n + 1 \in \Z, n \ge 2$
                \begin{enumerate}
                \item either $n + 1$ is prime
                \item or $n + 1$ is not prime i.e. $\exists d \in \N, d \ne 1 \wedge d = n + 1$
                \begin{align*}
                    d \mid n + 1 &\implies n + 1 = a \cdot d \\
                    \begin{aligned}
                        d \ne 1, n + 1 \\
                        a \ne 1, n + 1
                    \end{aligned} &\implies a, d < n + 1
                    & 2 \le a, d \le n
                \end{align*}
                \end{enumerate}
            \end{flushleft}
        \end{enumerate}
    \end{proof}
    \begin{theorem}
        A simple polygon with $n$ sides, where $n \in \{n \in \N \mid n \ge 3\}$, can be 
        triangulated into $n - 2$ triangles.
    \end{theorem}
    \begin{definition} \hfill
        \begin{enumerate}
            \item A \keyword{polygon} is a closed geometric figure consisting of a sequence of 
            line segments with $s_1, \dots, s_n$ as sides.
            \item A polygon is \keyword{simple} if no consecutive sides intersect.
            \item A polygon is \keyword{convex} if any line segment connecting 2 points in the 
            interior of the polygon lies inside the polygon.
            \item A \keyword{diagonal} is a line connecting 2 vertices of a polygon.
        \end{enumerate}
    \end{definition}
\end{document}
