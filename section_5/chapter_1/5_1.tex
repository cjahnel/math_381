\documentclass[letterpaper, 12pt]{article}
\usepackage{amsmath, amssymb}

\newtheorem{theorem}{Theorem}[section]
\newtheorem{lemma}[theorem]{Lemma}
\newtheorem{proposition}[theorem]{Proposition}
\newtheorem{corollary}[theorem]{Corollary}

\newenvironment{proof}[1][Proof]{\begin{trivlist}
\item[\hskip \labelsep {\bfseries #1}]}{\end{trivlist}}
\newenvironment{definition}[1][Definition]{\begin{trivlist}
\item[\hskip \labelsep {\bfseries #1}]}{\end{trivlist}}
\newenvironment{example}[1][Example]{\begin{trivlist}
\item[\hskip \labelsep {\bfseries #1}]}{\end{trivlist}}
\newenvironment{remark}[1][Remark]{\begin{trivlist}
\item[\hskip \labelsep {\bfseries #1}]}{\end{trivlist}}

\newcommand{\qed}{\quad \blacksquare}
\newcommand{\keyword}[1]{\textbf{#1}}
\newcommand{\then}{\rightarrow}
\newcommand{\bithen}{\leftrightarrow}

\newcommand{\N}{\mathbb{N}}
\newcommand{\Z}{\mathbb{Z}}
\newcommand{\Q}{\mathbb{Q}}
\newcommand{\R}{\mathbb{R}}
\newcommand{\C}{\mathbb{C}}
\newcommand{\power}{\mathcal{P}}
\newcommand{\0}{\emptyset}

\title{MATH 381 Section 5.1}
\author{Prof. Olivia Dumitrescu}
\date{23 February 2024}

\begin{document}
    \maketitle
    \subsection*{Gauss' Formula}
    \[1 + 2 + 3 + 4 + \dots + n = \frac{n(n + 1)}{2} = \binom{n+1}{2}\]
    \[1^2 + 2^2 + 3^2 + 4^2 + \dots + n^2 = \frac{n(n + 1)(2n + 1)}{6}\]
    \[1^3 + 2^3 + 3^3 + 4^3 + \dots + n^3 = \frac{n^2(n + 1)^2}{4}\]
    There is no forumla for an arbitrary power $k$.
    \subsection*{5.1.2 Mathematical Induction}
    Mathematical Induction can be used to prove statements such as $P(n)$ is true for all 
    positive integers.
    \subsection*{Proof of Mathematical Induction}
    \begin{enumerate}
        \item Basis step
        \[P(1) \equiv T\]
        \item Inductive step
        \[\forall k \in \N (P(k) \then P(k + 1))\]
    \end{enumerate}
    \begin{example}
        Prove Gauss' Formula
        \begin{proof}
            \begin{enumerate}
                \item \[P(1) = 1 = \frac{1 \cdot 2}{2} \equiv T\]
                $\therefore P(1) \equiv T$
                \item
                \begin{gather*}
                    P(k) = 1 + 2 + 3 + \dots + k = \frac{k(k + 1)}{2} \\
                    P(k + 1) = 1 + 2 + 3 + \dots + k + (k + 1) = \frac{k(k + 1)}{2} + (k + 1)
                \end{gather*}
                \begin{align*}
                    1 + 2 + 3 + \dots + k + (k + 1) &= \frac{k(k + 1)}{2} + (k + 1) \\
                    &= (k + 1)\left(\frac{k}{2} + 1\right) \\
                    &= (k + 1)\left(\frac{k}{2} + \frac{2}{2}\right) \\
                    &= (k + 1)\left(\frac{k + 2}{2}\right) \\
                    &= \frac{(k + 1)(k + 2)}{2} \\
                    &= \frac{(k + 1)((k + 1) + 1)}{2}
                \end{align*}
                $\therefore P(k) \then P(k + 1) \qed$
            \end{enumerate}
        \end{proof}
    \end{example}
    Francesco Maurolico (1494-1575) \\
    Let $S \subseteq \Z^+, S \ne \0$. \\
    By the well-ordering property, the subset $S$ must have a minimal element. 
    (i.e. $m \in S$ but $m-1 \notin S$) \\
    Mathematical Induction is equivalent to the well-ordering axiom.
    \begin{example}
        Prove
        \[P(n): 1^2 + 2^2 + 3^2 + \dots + n^2 = \frac{n(n + 1)(2n + 1)}{6}\]
    \end{example}
    \begin{proof}
        Mathematical Induction
        \begin{enumerate}
            \item $P(1)$
            \[1 = \frac{1(1 + 1)(2\cdot 1 + 1)}{6}\]
            \item Mathematical Induction \\
            Let $k \in \Z$ \\
            Assume $P(k)$ holds.
            \begin{align*}
                1^2 + 2^2 + \dots + k^2 &= \frac{k(k + 1)(2k + 1)}{6} \\
                1^2 + 2^2 + \dots + k^2 + (k + 1)^2 &= \frac{k(k + 1)(2k + 1)}{6} + (k + 1)^2 \\
                &= \frac{(k + 1)[k(2k + 1) + 6(k + 1)]}{6} \\
                &= \frac{(k + 1)[2k^2 + 7k + 6]}{6} \\
                &= \frac{(k + 1)(k + 2)(2k + 3)}{6} \\
                &= \frac{(k + 1)((k + 1) + 1)(2(k + 1) + 1)}{6}
            \end{align*}
            $\therefore P(k) \then P(k + 1) \qed$
        \end{enumerate}
    \end{proof}
    \begin{remark}
        Recall
        \[\int_{a}^{b} f(x) dx 
        = \lim_{n \to \infty} \frac{b - a}{n}\left[\sum_{i=1}^{n} f(x_i^*)\right]\]
    \end{remark}
    \begin{example}
        Compute
        \begin{align*}
            \int_{a=0}^{b=1} x^2 dx &= \lim_{n \to \infty} \frac{b-a}{n} \sum_{i=1}^{n} f(x_i^*) \\
            &= \lim_{n \to \infty} \frac{1}{n} (f(x_1^*) + f(x_2^*) + \dots + f(x_n^*)) \\
            &= \lim_{n \to \infty} \frac{1}{n} 
            \left(f\left(\frac{1}{n}\right) + f\left(\frac{2}{n}\right) + \dots + f\left(\frac{n}{n}\right)\right) \\
            &= \lim_{n \to \infty} \frac{1}{n} 
            \left(\left(\frac{1}{n}\right)^2 + \left(\frac{2}{n}\right)^2 + \dots + \left(\frac{n}{n}\right)^2\right) \\
            &= \lim_{n \to \infty} \frac{1}{n} \cdot \frac{1^2 + 2^2 + 3^2 + \dots + n^2}{n^2} \\
            &= \lim_{n \to \infty} \frac{1}{n} \cdot \frac{\dfrac{n(n+1)(2n+1)}{6}}{n^2} \\
            &= \lim_{n \to \infty} \frac{n(n+1)(2n+1)}{6n^3} \\
            &\sim \frac{2n^3}{6n^3} = \frac{1}{3}
        \end{align*}
    \end{example}
    \begin{example}
        Use Mathematical Induction to prove $7^{n + 2} + 8^{2n + 1}$ is divisible by 57 for 
        any non-negative integer $n$.
        \[P(k): 57 \mid (7^{n + 2} + 8^{2n + 1})\]
    \end{example}
    \begin{proof}
        \begin{enumerate}
            \item Basis step
            \begin{align*}
                P(1): 57 &\mid (7^{1 + 2} + 8^{2 \cdot 1 + 1}) \\
                P(1): 57 &\mid (343 + 512) \\
                P(1): 57 &\mid 855 \equiv T
            \end{align*}
            $\therefore P(1)$
            \item Inductive step \\
            Let $k \in \N$.
            \begin{align*}
                &\quad\; 7^{n + 3} + 8^{2n + 3} \\
                &= 7 \cdot 7^{n + 2} + 8^2 \cdot 8^{2n + 1} \\
                &= 7 \cdot 7^{n + 2} + (57 + 7) \cdot 8^{2n + 1} \\
                &= 7 \cdot 7^{n + 2} + 7 \cdot 8^{2n + 1} + 57 \cdot 8^{2n + 1}
            \end{align*}
            \[57 \mid 7(7^{n + 2} + 8^{2n + 1}) \wedge 57 \mid 57 \cdot 8^{2n + 1} 
            \implies 57 \mid (7 \cdot 7^{n + 2} + 7 \cdot 8^{2n + 1} + 57 \cdot 8^{2n + 1})\]
            $\therefore P(k) \then P(k + 1) \qed$
        \end{enumerate}
    \end{proof}
    \begin{example}
        What is the sum of all positive odd integers?
        \[P(k): 1 + 3 + 5 + \dots + (2k - 1) = k^2\]
    \end{example}
    \begin{proof}
        \begin{enumerate}
            \item Basis step
            \[1 = 1\]
            $\therefore P(1)$
            \item Inductive step \\
            Assume $P(k)$, $k \in \N$.
            \begin{align*}
                1 + 3 + 5 + \dots + (2k - 1) &= k^2 \\
                1 + 3 + 5 + \dots + (2k - 1) + (2k + 1) &= k^2 + (2k + 1) \\
                &= (k + 1)^2
            \end{align*}
            $\therefore P(k) \then P(k + 1) \qed$
        \end{enumerate}
    \end{proof}
    \begin{example}
        Use mathematical induction to prove 
        \[P(n): 1 + r + r^2 + \dots + r^n = \frac{r^{n+1} - 1}{r - 1} \qquad r \ne 1\]
    \end{example}
    \begin{proof}
        \begin{enumerate}
            \item $P(1): 1 + r = \dfrac{r^2 - 1}{r - 1} \equiv T$ \\
            $\therefore P(1)$
            \item \begin{align*}
                P(n) + r^{n+1} = 1 + r + r^2 + \dots + r^n &= \frac{r^{n+1} - 1}{r - 1} + r^{n+1} \\
                &= \frac{r^{n+1} - 1}{r - 1} + \frac{r^{n+1}(r-1)}{r-1} \\
                &= \frac{r^{n+1} - 1 + r^{n+1}(r-1)}{r-1} \\
                &= \frac{r^{n+1} - 1 + r^{n+2} - r^{n+1}}{r-1} \\
                &= \frac{r^{n+2} - 1}{r-1} \\
                &= P(n + 1)
            \end{align*}
            $\therefore P(n) \then P(n + 1) \qed$
        \end{enumerate}
    \end{proof}
    \begin{theorem}
        If $S$ is a finite set, $|\power(S)| = 2^{|S|}$.
    \end{theorem}
    \begin{proof}
        \begin{enumerate}
            \item Let $S = \{y\}$ \\
            Subsets of S = $\begin{cases}
                \0 \\
                \{y\}
            \end{cases} \implies |\power(S)| = 2 = 2^{|S|}$ \\
            $\therefore P(1)$
            \item \[
                S = \{x_1, \dots, x_n, A\} \qquad |S| = n + 1
            \]
            $\alpha$: Subsets of $S$ that do not include $A$ are subsets of 
            $\{x_1, \dots, x_n, A\}$. We have $2^n$. \\
            $\beta$: Subsets of $S$ that include $A$ are subsets of $\{x_1, \dots, x_n, A\}$ 
            with an $A$ appended to it. Therefore, we have another $2^n$.
            \[\alpha + \beta = 2^n +2^n = 2^{n+1}\]
            $\therefore P(n) \then P(n + 1) \qed$
        \end{enumerate}
    \end{proof}
    \begin{example}
        Denote $H_j = 1 + \frac{1}{2} + \frac{1}{3} + \dots + \frac{1}{j}$. \\
        Use mathematical induction to prove 
        \[H_{2^n} \ge 1 + \frac{n}{2}\]
        for any non-negative integer $n$.
    \end{example}
    \begin{proof} Induction on $n$
        \begin{enumerate}
            \item \[
                P(1): 1 + \frac{1}{2} \ge 1 + \frac{1}{2}
            \]
            $\therefore P(1)$
            \item \begin{align*}
                1 + \frac{1}{2} + \frac{1}{3} + \dots + \frac{1}{2^n} &\ge 1 + \frac{n}{2} \\
                (1 + \frac{1}{2} + \frac{1}{3} + \dots + \frac{1}{2^n}) 
                + (\frac{1}{2^n + 1} + \frac{1}{2^n + 2} + \dots + \frac{1}{2^{n+1}}) 
                &\ge 1 + \frac{n + 1}{2}
            \end{align*}
            \begin{multline*}
                1 + \frac{1}{2} + \frac{1}{3} + \dots + \frac{1}{2^n} 
                + (\frac{1}{2^n + 1} + \frac{1}{2^n + 2} + \dots + \frac{1}{2^{n+1}}) \\
                \ge 1 + \frac{n}{2} 
                + (\frac{1}{2^n + 1} + \frac{1}{2^n + 2} + \dots + \frac{1}{2^{n+1}}) 
            \end{multline*}
            \begin{align*}
                1 + \frac{n}{2} 
                + (\frac{1}{2^n + 1} + \frac{1}{2^n + 2} + \dots + \frac{1}{2^{n+1}}) 
                \ge 1 + \frac{n + 1}{2} \\
                \frac{1}{2^n + 1} + \frac{1}{2^n + 2} + \dots + \frac{1}{2^{n+1}} 
                \ge \frac{1}{2} \\
                \frac{1}{2^n + 1} + \frac{1}{2^n + 2} + \dots + \frac{1}{2^n + 2^n} 
                \ge \frac{1}{2^{n+1}} + \frac{1}{2^{n+1}} + \dots + \frac{1}{2^{n+1}}
                = \frac{2^n}{2^{n+1}} = \frac{1}{2}
            \end{align*}
            $\therefore P(n) \then P(n+1) \qed$
        \end{enumerate}
    \end{proof}
\end{document}
